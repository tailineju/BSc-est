% Options for packages loaded elsewhere
\PassOptionsToPackage{unicode}{hyperref}
\PassOptionsToPackage{hyphens}{url}
%
\documentclass[
]{article}
\usepackage{amsmath,amssymb}
\usepackage{lmodern}
\usepackage{iftex}
\ifPDFTeX
  \usepackage[T1]{fontenc}
  \usepackage[utf8]{inputenc}
  \usepackage{textcomp} % provide euro and other symbols
\else % if luatex or xetex
  \usepackage{unicode-math}
  \defaultfontfeatures{Scale=MatchLowercase}
  \defaultfontfeatures[\rmfamily]{Ligatures=TeX,Scale=1}
\fi
% Use upquote if available, for straight quotes in verbatim environments
\IfFileExists{upquote.sty}{\usepackage{upquote}}{}
\IfFileExists{microtype.sty}{% use microtype if available
  \usepackage[]{microtype}
  \UseMicrotypeSet[protrusion]{basicmath} % disable protrusion for tt fonts
}{}
\makeatletter
\@ifundefined{KOMAClassName}{% if non-KOMA class
  \IfFileExists{parskip.sty}{%
    \usepackage{parskip}
  }{% else
    \setlength{\parindent}{0pt}
    \setlength{\parskip}{6pt plus 2pt minus 1pt}}
}{% if KOMA class
  \KOMAoptions{parskip=half}}
\makeatother
\usepackage{xcolor}
\IfFileExists{xurl.sty}{\usepackage{xurl}}{} % add URL line breaks if available
\IfFileExists{bookmark.sty}{\usepackage{bookmark}}{\usepackage{hyperref}}
\hypersetup{
  pdftitle={P1 antiga},
  pdfauthor={César A. Galvão - 19/0011572},
  hidelinks,
  pdfcreator={LaTeX via pandoc}}
\urlstyle{same} % disable monospaced font for URLs
\usepackage[margin=1in]{geometry}
\usepackage{graphicx}
\makeatletter
\def\maxwidth{\ifdim\Gin@nat@width>\linewidth\linewidth\else\Gin@nat@width\fi}
\def\maxheight{\ifdim\Gin@nat@height>\textheight\textheight\else\Gin@nat@height\fi}
\makeatother
% Scale images if necessary, so that they will not overflow the page
% margins by default, and it is still possible to overwrite the defaults
% using explicit options in \includegraphics[width, height, ...]{}
\setkeys{Gin}{width=\maxwidth,height=\maxheight,keepaspectratio}
% Set default figure placement to htbp
\makeatletter
\def\fps@figure{htbp}
\makeatother
\setlength{\emergencystretch}{3em} % prevent overfull lines
\providecommand{\tightlist}{%
  \setlength{\itemsep}{0pt}\setlength{\parskip}{0pt}}
\setcounter{secnumdepth}{5}
\usepackage{helvet} \renewcommand\familydefault{\sfdefault}
\usepackage{booktabs}
\usepackage{longtable}
\usepackage{array}
\usepackage{multirow}
\usepackage{wrapfig}
\usepackage{float}
\usepackage{colortbl}
\usepackage{pdflscape}
\usepackage{tabu}
\usepackage{threeparttable}
\usepackage{threeparttablex}
\usepackage[normalem]{ulem}
\usepackage{makecell}
\usepackage{xcolor}
\ifLuaTeX
  \usepackage{selnolig}  % disable illegal ligatures
\fi

\title{P1 antiga}
\author{César A. Galvão - 19/0011572}
\date{2022-07-08}

\begin{document}
\maketitle

\newpage{}

{
\setcounter{tocdepth}{2}
\tableofcontents
}
\let\oldsection\section
\renewcommand\section{\clearpage\oldsection}

\hypertarget{questao-1}{%
\section{Questao 1}\label{questao-1}}

\begin{longtable}{cccc}
\toprule
obs & 20g & 30g & 40g\\
\midrule
\endfirsthead
\multicolumn{4}{@{}l}{\textit{(continued)}}\\
\toprule
obs & 20g & 30g & 40g\\
\midrule
\endhead

\endfoot
\bottomrule
\endlastfoot
\cellcolor{gray!15}{obs1} & \cellcolor{gray!15}{24} & \cellcolor{gray!15}{37} & \cellcolor{gray!15}{42}\\
obs2 & 28 & 44 & 47\\
\cellcolor{gray!15}{obs3} & \cellcolor{gray!15}{37} & \cellcolor{gray!15}{39} & \cellcolor{gray!15}{52}\\
obs4 & 30 & 35 & 38\\*
\end{longtable}

\hypertarget{modelo}{%
\subsection{Modelo}\label{modelo}}

O modelo escolhido para a avaliação dos tratamentos é o modelo de
efeitos, expresso na equação a seguir

\begin{equation}
  y_{ij} = \mu + \tau_i + e_{ij}, \quad i = 1, 2,..., a; \quad j = 1, 2,..., n
\end{equation}

em que \(\mu\) é a média geral, \(\tau_i\) é a média ou efeito dos
grupos e \(e_{ij}\) é o desvio do elemento. Os grupos são indexados por
\(i\) e os indivíduos de cada grupo indexados por \(j\).

Considera-se para utilização do modelo de efeitos:

\begin{itemize}
\tightlist
\item
  independência entre realizações dos testes;
\item
  normalidade de distribuição dos resíduos;
\item
  homogeneidade de variâncias (homocedasticidade) dos resíduos.
\end{itemize}

Estima-se, considerando \(\bar{x}\) o estimador natural para \(\mu\) e
QMRES \(= \hat{\sigma}^2\):

\begin{longtable}{ccccc}
\toprule
$\bar{x}$ & $\tau_1$ & $\tau_2$ & $\tau_3$ & $\hat{\sigma}^2$\\
\midrule
\endfirsthead
\multicolumn{5}{@{}l}{\textit{(continued)}}\\
\toprule
$\bar{x}$ & $\tau_1$ & $\tau_2$ & $\tau_3$ & $\hat{\sigma}^2$\\
\midrule
\endhead

\endfoot
\bottomrule
\endlastfoot
\cellcolor{gray!15}{37.75} & \cellcolor{gray!15}{29.75} & \cellcolor{gray!15}{38.75} & \cellcolor{gray!15}{44.75} & \cellcolor{gray!15}{27.14}\\*
\end{longtable}

\hypertarget{anova}{%
\subsection{ANOVA}\label{anova}}

As hipóteses do teste de análise de variância são as seguintes:
\begin{align}
  \begin{cases}
    H_0: \tau_1 = ... = \tau_a = 0, \quad \text{(O efeito de tratamento é nulo)}\\
    H_1: \exists \tau_i \neq 0
  \end{cases}
\end{align}

que equivale dizer

\begin{align}
  \begin{cases}
    H_0: \mu_1 = ... = \mu_a\\
    H_1: \exists \mu_i \neq \mu_j, \, i \neq j.
  \end{cases}
\end{align}

A tabela de análise de variâncias é apresentada a seguir:

\begin{longtable}{cccccc}
\toprule
term & df & sumsq & meansq & statistic & p.value\\
\midrule
\endfirsthead
\multicolumn{6}{@{}l}{\textit{(continued)}}\\
\toprule
term & df & sumsq & meansq & statistic & p.value\\
\midrule
\endhead

\endfoot
\bottomrule
\endlastfoot
\cellcolor{gray!15}{dosagem} & \cellcolor{gray!15}{2} & \cellcolor{gray!15}{456.00} & \cellcolor{gray!15}{228.00000} & \cellcolor{gray!15}{8.401228} & \cellcolor{gray!15}{0.0087421}\\
Residuals & 9 & 244.25 & 27.13889 & NA & NA\\*
\end{longtable}

Considerando \(\alpha = 0,05\), rejeita-se a hipótese nula. Isto é,
rejeita-se a hipótese de que há igualdade entre as médias de cada
tratamento.

De fato, realizando testes diagnósticos para normalidade e igualdade de
variâncias, não se rejeita a hipótese de normalidade dos resíduos e
considera-se a variância igual entre grupos, conforme apresentado na
tabela a seguir:

\begin{longtable}{ccc}
\toprule
Estatística de teste & p-valor & Método\\
\midrule
\endfirsthead
\multicolumn{3}{@{}l}{\textit{(continued)}}\\
\toprule
Estatística de teste & p-valor & Método\\
\midrule
\endhead

\endfoot
\bottomrule
\endlastfoot
\cellcolor{gray!15}{0.9382542} & \cellcolor{gray!15}{0.4757921} & \cellcolor{gray!15}{Shapiro-Wilk normality test}\\
0.4982699 & 0.6234012 & Levene igual. vars.\\*
\end{longtable}

\hypertarget{teste-de-fisher}{%
\subsection{Teste de Fisher}\label{teste-de-fisher}}

O teste é realizado utilizando as estatísticas de teste \(t_0\)

\begin{align}
  t_0 = \frac{|\bar{y}_{i.} - \bar{y}_{j.}|}{\sqrt{\text{QMRES} \left( \frac{1}{n_i}+\frac{1}{n_j} \right)}}
\end{align}

com \(an-a\) graus de liberdade sob a hipótese nula. Seus p-valores são
apresentados na tabela a seguir:

\begin{longtable}{ccc}
\toprule
Grupo.1 & Grupo.2 & p.valores\\
\midrule
\endfirsthead
\multicolumn{3}{@{}l}{\textit{(continued)}}\\
\toprule
Grupo.1 & Grupo.2 & p.valores\\
\midrule
\endhead

\endfoot
\bottomrule
\endlastfoot
\cellcolor{gray!15}{20g} & \cellcolor{gray!15}{30g} & \cellcolor{gray!15}{0.0371682}\\
20g & 40g & 0.0027912\\
\cellcolor{gray!15}{30g} & \cellcolor{gray!15}{40g} & \cellcolor{gray!15}{0.1377938}\\*
\end{longtable}

Considerando \(\alpha = 0,05\) se rejeita a hipótese de igualdade entre
as médias do grupo de 20g com o grupo 30g pelo teste de Fisher.

\hypertarget{teste-de-tukey}{%
\subsection{Teste de Tukey}\label{teste-de-tukey}}

\begin{longtable}{cc}
\toprule
Comparações & p.valores\\
\midrule
\endfirsthead
\multicolumn{2}{@{}l}{\textit{(continued)}}\\
\toprule
Comparações & p.valores\\
\midrule
\endhead

\endfoot
\bottomrule
\endlastfoot
\cellcolor{gray!15}{30g-20g} & \cellcolor{gray!15}{0.0859425}\\
40g-20g & 0.0070702\\
\cellcolor{gray!15}{40g-30g} & \cellcolor{gray!15}{0.2833753}\\*
\end{longtable}

Pelo teste de Tukey, apenas se rejeita a hipótese de igualdade entre as
médias dos grupos de 40g e 20g, com o mesmo nível de confiança.

\hypertarget{maximizauxe7uxe3o-da-bioatividade}{%
\subsection{Maximização da
bioatividade}\label{maximizauxe7uxe3o-da-bioatividade}}

Considerando que pelo teste de Tukey os grupos de 20g e 40g têm médias
estatisticamente diferentes e que este apresenta maior bioatividade,
opta-se pelo tratamento de 40g. O intervalo de confiança é construído da
seguinte forma:

\begin{align}
  IC(\bar{x}; 0,9) &= \bar{x} \pm t_{(an-a; 1-\alpha/2)} \cdot \sqrt{\frac{\text{QMRES}}{n}} \\
  &= 44,75 \pm 1,83 \cdot  2,61 \\
  &= \left[39,9737; 49,5263  \right]
\end{align}

\hypertarget{calcule-e22}{%
\subsection{Calcule e22}\label{calcule-e22}}

\begin{align}
  y_{22} &= \mu + \tau_2 + e_{22}\\
  44 &= 37.75 + 1 + e_{22} \\
  e_{22} &= 5.25
\end{align}

\hypertarget{erro-tipo-ii}{%
\subsection{Erro tipo II}\label{erro-tipo-ii}}

\begin{align}
  P\left( F_{\text{obs}} < F_\text{crit} \bigg| \phi^2 =  \frac{n}{\sigma^2} \sum\limits_{i=1}^{3} \tau_i^2  \right),
\end{align}

considerando a variância para os resíduos. Portanto,

\begin{align}
  \phi^2 &= \frac{n}{\text{QMRES}} 30 \\
  &= \frac{3}{27,139} 30 \\
  &= 4,42
\end{align}

é o parâmetro de não-centralidade (pnc ou, em inglês, \emph{ncp}) da
distribuição \(F\) e, sob \(H_0\), \(\phi^2 = 0\).

O valor
\(F_\text{crit} = F( \gamma = 0,95; gl_1 = 2; gl_2 = 9, \phi^2 = 0)\) é
de 4.256. Considerando \(\phi^2 =\) 4.4217, obtém-se

\begin{align}
  P\left( F_{\text{obs}} < F_\text{crit} \big| \phi^2 \text{ sob } H_1 \right) &= P\left( F_{\text{obs}} < 4,256 \big| \phi^2 = 4,422 \right) \\
  &= 0,662.
\end{align}

\hypertarget{probabilidade-erro-tipo-ii-inferior-a-50}\label{probabilidade-erro-tipo-ii-inferior-a-50}}

Considerando os métodos de cálculo já utilizados, constroi-se a tabela a
seguir:

\begin{longtable}{ccccccc}
\toprule
n & $\phi^2$ & $\phi$ & g.l. & $F_{\text{crit}}$ & $\beta$ & Poder\\
\midrule
\endfirsthead
\multicolumn{7}{@{}l}{\textit{(continued)}}\\
\toprule
n & $\phi^2$ & $\phi$ & g.l. & $F_{\text{crit}}$ & $\beta$ & Poder\\
\midrule
\endhead

\endfoot
\bottomrule
\endlastfoot
\cellcolor{gray!15}{4} & \cellcolor{gray!15}{4.42} & \cellcolor{gray!15}{2.10} & \cellcolor{gray!15}{9} & \cellcolor{gray!15}{4.26} & \cellcolor{gray!15}{0.66} & \cellcolor{gray!15}{0.34}\\
5 & 5.53 & 2.35 & 12 & 3.89 & 0.56 & 0.44\\
\cellcolor{gray!15}{6} & \cellcolor{gray!15}{6.63} & \cellcolor{gray!15}{2.58} & \cellcolor{gray!15}{15} & \cellcolor{gray!15}{3.68} & \cellcolor{gray!15}{0.46} & \cellcolor{gray!15}{0.54}\\*
\end{longtable}

Considerando os valores da tabela, para que o erro tipo II seja menor
que 50\% são necessários 6 repetições para cada tratamento nesse
experimento.

\hypertarget{questuxe3o-2}{%
\section{Questão 2}\label{questuxe3o-2}}

Na tabela, \(D1\) representa a diferença entre as datas 2 e 1 e \(D2\) o
mesmo para datas 4 e 3.

\begin{longtable}{ccccccc}
\toprule
Árvores & Data 1 & Data 2 & Data 3 & Data 4 & D1 & D2\\
\midrule
\endfirsthead
\multicolumn{7}{@{}l}{\textit{(continued)}}\\
\toprule
Árvores & Data 1 & Data 2 & Data 3 & Data 4 & D1 & D2\\
\midrule
\endhead

\endfoot
\bottomrule
\endlastfoot
\cellcolor{gray!15}{arvore 1} & \cellcolor{gray!15}{30} & \cellcolor{gray!15}{58} & \cellcolor{gray!15}{115} & \cellcolor{gray!15}{120} & \cellcolor{gray!15}{28} & \cellcolor{gray!15}{5}\\
arvore 2 & 33 & 69 & 156 & 172 & 36 & 16\\
\cellcolor{gray!15}{arvore 3} & \cellcolor{gray!15}{30} & \cellcolor{gray!15}{51} & \cellcolor{gray!15}{108} & \cellcolor{gray!15}{115} & \cellcolor{gray!15}{21} & \cellcolor{gray!15}{7}\\
arvore 4 & 32 & 62 & 167 & 179 & 30 & 12\\
\cellcolor{gray!15}{arvore 5} & \cellcolor{gray!15}{30} & \cellcolor{gray!15}{49} & \cellcolor{gray!15}{125} & \cellcolor{gray!15}{142} & \cellcolor{gray!15}{19} & \cellcolor{gray!15}{17}\\*
\end{longtable}

Deseja-se testar se o crescimento entre os períodos é diferente.
Interessa testar portanto se as médias de \(D1\) e \(D2\) são
diferentes. Antes de realizar um teste de comparaçao de médias para
medidas repetidas, i.e.~um teste \(t\) pareado, testa-se para igualdade
de variâncias.

Considerando \(H_0\) sendo a igualdade entre as variâncias, obtém-se
p-valor 0.625. Ou seja, não se rejeita a hipótese de igualdade de
variâncias.

Tomando a hipótese de normalidade dos dados como verdadeira, realiza-se
o teste \(t\) pareado, para variâncias iguais, com nível de confiança
\(\gamma = 0,95\). A hipótese nula do teste é a igualdade entre as
médias, de modo que o teste realizado é bilateral.

O p-valor obtido no teste é de 0.014, de modo que é possível rejeitar a
hipótese nula a \(\alpha = 0,05\). Ou seja, pode-se dizer que há uma
diferença estatisticamente diferente entre os crescimentos nos dois
períodos.

\end{document}
