% Options for packages loaded elsewhere
\PassOptionsToPackage{unicode}{hyperref}
\PassOptionsToPackage{hyphens}{url}
%
\documentclass[
]{article}
\usepackage{amsmath,amssymb}
\usepackage{lmodern}
\usepackage{iftex}
\ifPDFTeX
  \usepackage[T1]{fontenc}
  \usepackage[utf8]{inputenc}
  \usepackage{textcomp} % provide euro and other symbols
\else % if luatex or xetex
  \usepackage{unicode-math}
  \defaultfontfeatures{Scale=MatchLowercase}
  \defaultfontfeatures[\rmfamily]{Ligatures=TeX,Scale=1}
\fi
% Use upquote if available, for straight quotes in verbatim environments
\IfFileExists{upquote.sty}{\usepackage{upquote}}{}
\IfFileExists{microtype.sty}{% use microtype if available
  \usepackage[]{microtype}
  \UseMicrotypeSet[protrusion]{basicmath} % disable protrusion for tt fonts
}{}
\makeatletter
\@ifundefined{KOMAClassName}{% if non-KOMA class
  \IfFileExists{parskip.sty}{%
    \usepackage{parskip}
  }{% else
    \setlength{\parindent}{0pt}
    \setlength{\parskip}{6pt plus 2pt minus 1pt}}
}{% if KOMA class
  \KOMAoptions{parskip=half}}
\makeatother
\usepackage{xcolor}
\IfFileExists{xurl.sty}{\usepackage{xurl}}{} % add URL line breaks if available
\IfFileExists{bookmark.sty}{\usepackage{bookmark}}{\usepackage{hyperref}}
\hypersetup{
  pdftitle={Exercício de laboratório 4},
  pdfauthor={César A. Galvão - 19/0011572},
  hidelinks,
  pdfcreator={LaTeX via pandoc}}
\urlstyle{same} % disable monospaced font for URLs
\usepackage[margin=1in]{geometry}
\usepackage{graphicx}
\makeatletter
\def\maxwidth{\ifdim\Gin@nat@width>\linewidth\linewidth\else\Gin@nat@width\fi}
\def\maxheight{\ifdim\Gin@nat@height>\textheight\textheight\else\Gin@nat@height\fi}
\makeatother
% Scale images if necessary, so that they will not overflow the page
% margins by default, and it is still possible to overwrite the defaults
% using explicit options in \includegraphics[width, height, ...]{}
\setkeys{Gin}{width=\maxwidth,height=\maxheight,keepaspectratio}
% Set default figure placement to htbp
\makeatletter
\def\fps@figure{htbp}
\makeatother
\setlength{\emergencystretch}{3em} % prevent overfull lines
\providecommand{\tightlist}{%
  \setlength{\itemsep}{0pt}\setlength{\parskip}{0pt}}
\setcounter{secnumdepth}{5}
\usepackage{helvet} \renewcommand\familydefault{\sfdefault}
\usepackage{booktabs}
\usepackage{longtable}
\usepackage{array}
\usepackage{multirow}
\usepackage{wrapfig}
\usepackage{float}
\usepackage{colortbl}
\usepackage{pdflscape}
\usepackage{tabu}
\usepackage{threeparttable}
\usepackage{threeparttablex}
\usepackage[normalem]{ulem}
\usepackage{makecell}
\usepackage{xcolor}
\ifLuaTeX
  \usepackage{selnolig}  % disable illegal ligatures
\fi

\title{Exercício de laboratório 4}
\usepackage{etoolbox}
\makeatletter
\providecommand{\subtitle}[1]{% add subtitle to \maketitle
  \apptocmd{\@title}{\par {\large #1 \par}}{}{}
}
\makeatother
\subtitle{Blocos completos}
\author{César A. Galvão - 19/0011572}
\date{2022-08-10}

\begin{document}
\maketitle

\newpage{}

{
\setcounter{tocdepth}{2}
\tableofcontents
}
\let\oldsection\section
\renewcommand\section{\clearpage\oldsection}

\hypertarget{questao-1}{%
\section{Questao 1}\label{questao-1}}

\begin{longtable}{cccccc}
\toprule
Químico & Bolt 1 & Bolt 2 & Bolt3 & Bolt 4 & Bolt 5\\
\midrule
\endfirsthead
\multicolumn{6}{@{}l}{\textit{(continued)}}\\
\toprule
Químico & Bolt 1 & Bolt 2 & Bolt3 & Bolt 4 & Bolt 5\\
\midrule
\endhead

\endfoot
\bottomrule
\endlastfoot
\cellcolor{gray!15}{1} & \cellcolor{gray!15}{73} & \cellcolor{gray!15}{68} & \cellcolor{gray!15}{74} & \cellcolor{gray!15}{71} & \cellcolor{gray!15}{67}\\
2 & 73 & 67 & 75 & 72 & 70\\
\cellcolor{gray!15}{3} & \cellcolor{gray!15}{75} & \cellcolor{gray!15}{68} & \cellcolor{gray!15}{78} & \cellcolor{gray!15}{73} & \cellcolor{gray!15}{68}\\
4 & 73 & 71 & 75 & 75 & 69\\*
\end{longtable}

\hypertarget{modelo-e-hipuxf3teses}{%
\subsection{Modelo e hipóteses}\label{modelo-e-hipuxf3teses}}

É utilizado o RCBD, \emph{randomized complete block design},
representado por:

\begin{align*}
  y_{ij} = \mu + \tau_i + \beta_j + e_{ij}, \quad i = 1, 2,..., a; \quad j = 1, 2,..., n
\end{align*}

em que \(\mu\) é a média geral, \(\tau_i\) é a média ou efeito dos
grupos -- cada químico sendo considerado um tratamento --, \(\beta_j\) é
a bloco -- aqui cada \emph{bolt} ou o equivalente a lote -- e \(e_{ij}\)
é o desvio do elemento. Os grupos são indexados por \(i\) e os blocos
indexados por \(j\).

As hipóteses do teste são as seguintes: \begin{align*}
  \begin{cases}
    H_0: \tau_1 = ... = \tau_a = 0, \quad \text{(O efeito de tratamento é nulo)}\\
    H_1: \exists \tau_i \neq 0
  \end{cases}
\end{align*}

que equivale dizer

\begin{align*}
  \begin{cases}
    H_0: \mu_1 = ... = \mu_a\\
    H_1: \exists \mu_i \neq \mu_j, \, i \neq j.
  \end{cases}
\end{align*}

Mesmo que o interesse do estudo não seja sobre o efeito dos blocos, é
interessante testá-los para avaliar se é necessário manter a estrutura
de blocos e futuras replicações do experimento.

\hypertarget{tabela-de-anova}{%
\subsection{Tabela de ANOVA}\label{tabela-de-anova}}

De acordo com a tabela a seguir, de fato os blocos apresentam efeito
significativo sobre a variância do experimento, o que não ocorre para os
tratamentos, evidenciado pelo p-valor de 0.121.

\begin{longtable}{lccccl}
\toprule
Fonte de variação & g.l. & SQ & MQ & Estatística F & p-valor\\
\midrule
\endfirsthead
\multicolumn{6}{@{}l}{\textit{(continued)}}\\
\toprule
Fonte de variação & g.l. & SQ & MQ & Estatística F & p-valor\\
\midrule
\endhead

\endfoot
\bottomrule
\endlastfoot
\cellcolor{gray!15}{bolts} & \cellcolor{gray!15}{4} & \cellcolor{gray!15}{157.00} & \cellcolor{gray!15}{39.2500} & \cellcolor{gray!15}{21.6055} & \cellcolor{gray!15}{0.0000}\\
chem & 3 & 12.95 & 4.3167 & 2.3761 & 0.1211\\
\cellcolor{gray!15}{Residuals} & \cellcolor{gray!15}{12} & \cellcolor{gray!15}{21.80} & \cellcolor{gray!15}{1.8167} & \cellcolor{gray!15}{NA} & \cellcolor{gray!15}{NA}\\*
\end{longtable}

Mediante realização do teste Shapiro para normalidade, obtém-se p-valor
de 0.041. Como ANOVA é um teste paramétrico, devemos utilizar outro
teste para a avaliação da diferença estatística entre os tratamentos.

\hypertarget{estimativa-dos-paruxe2metros-do-modelo}{%
\subsection{Estimativa dos parâmetros do
modelo}\label{estimativa-dos-paruxe2metros-do-modelo}}

\begin{longtable}{cc}
\toprule
$\mu$ & $\sigma^2$\\
\midrule
\endfirsthead
\multicolumn{2}{@{}l}{\textit{(continued)}}\\
\toprule
$\mu$ & $\sigma^2$\\
\midrule
\endhead

\endfoot
\bottomrule
\endlastfoot
\cellcolor{gray!15}{71.75} & \cellcolor{gray!15}{1.82}\\*
\end{longtable}

\begin{longtable}{cccc}
\toprule
$\tau_1$ & $\tau_2$ & $\tau_3$ & $\tau_4$\\
\midrule
\endfirsthead
\multicolumn{4}{@{}l}{\textit{(continued)}}\\
\toprule
$\tau_1$ & $\tau_2$ & $\tau_3$ & $\tau_4$\\
\midrule
\endhead

\endfoot
\bottomrule
\endlastfoot
\cellcolor{gray!15}{-1.15} & \cellcolor{gray!15}{-0.35} & \cellcolor{gray!15}{0.65} & \cellcolor{gray!15}{0.85}\\*
\end{longtable}

\begin{longtable}{ccccc}
\toprule
$\beta_1$ & $\beta_2$ & $\beta_3$ & $\beta_4$ & $\beta_5$\\
\midrule
\endfirsthead
\multicolumn{5}{@{}l}{\textit{(continued)}}\\
\toprule
$\beta_1$ & $\beta_2$ & $\beta_3$ & $\beta_4$ & $\beta_5$\\
\midrule
\endhead

\endfoot
\bottomrule
\endlastfoot
\cellcolor{gray!15}{1.75} & \cellcolor{gray!15}{-3.25} & \cellcolor{gray!15}{3.75} & \cellcolor{gray!15}{1} & \cellcolor{gray!15}{-3.25}\\*
\end{longtable}

\hypertarget{qual-elemento-quuxedmico-deve-ser-recomendado}{%
\subsection{Qual elemento químico deve ser
recomendado?}\label{qual-elemento-quuxedmico-deve-ser-recomendado}}

Considerando apenas a análise de variância, tanto faz o elemento
utilizado, já que não há diferença entre tratamentos. Procede-se ao
teste de Tukey para avaliação de pares de tratamentos.

\begin{longtable}{lcccc}
\toprule
  & diff & lwr & upr & p adj\\
\midrule
\endfirsthead
\multicolumn{5}{@{}l}{\textit{(continued)}}\\
\toprule
  & diff & lwr & upr & p adj\\
\midrule
\endhead

\endfoot
\bottomrule
\endlastfoot
\cellcolor{gray!15}{b2-b1} & \cellcolor{gray!15}{-5.00} & \cellcolor{gray!15}{-8.04} & \cellcolor{gray!15}{-1.96} & \cellcolor{gray!15}{0.00}\\
b3-b1 & 2.00 & -1.04 & 5.04 & 0.28\\
\cellcolor{gray!15}{b4-b1} & \cellcolor{gray!15}{-0.75} & \cellcolor{gray!15}{-3.79} & \cellcolor{gray!15}{2.29} & \cellcolor{gray!15}{0.93}\\
b5-b1 & -5.00 & -8.04 & -1.96 & 0.00\\
\cellcolor{gray!15}{b3-b2} & \cellcolor{gray!15}{7.00} & \cellcolor{gray!15}{3.96} & \cellcolor{gray!15}{10.04} & \cellcolor{gray!15}{0.00}\\
b4-b2 & 4.25 & 1.21 & 7.29 & 0.01\\
\cellcolor{gray!15}{b5-b2} & \cellcolor{gray!15}{0.00} & \cellcolor{gray!15}{-3.04} & \cellcolor{gray!15}{3.04} & \cellcolor{gray!15}{1.00}\\
b4-b3 & -2.75 & -5.79 & 0.29 & 0.08\\
\cellcolor{gray!15}{b5-b3} & \cellcolor{gray!15}{-7.00} & \cellcolor{gray!15}{-10.04} & \cellcolor{gray!15}{-3.96} & \cellcolor{gray!15}{0.00}\\
b5-b4 & -4.25 & -7.29 & -1.21 & 0.01\\*
\end{longtable}

\begin{longtable}{lcccc}
\toprule
  & diff & lwr & upr & p adj\\
\midrule
\endfirsthead
\multicolumn{5}{@{}l}{\textit{(continued)}}\\
\toprule
  & diff & lwr & upr & p adj\\
\midrule
\endhead

\endfoot
\bottomrule
\endlastfoot
\cellcolor{gray!15}{2-1} & \cellcolor{gray!15}{0.8} & \cellcolor{gray!15}{-1.73} & \cellcolor{gray!15}{3.33} & \cellcolor{gray!15}{0.79}\\
3-1 & 1.8 & -0.73 & 4.33 & 0.20\\
\cellcolor{gray!15}{4-1} & \cellcolor{gray!15}{2.0} & \cellcolor{gray!15}{-0.53} & \cellcolor{gray!15}{4.53} & \cellcolor{gray!15}{0.14}\\
3-2 & 1.0 & -1.53 & 3.53 & 0.65\\
\cellcolor{gray!15}{4-2} & \cellcolor{gray!15}{1.2} & \cellcolor{gray!15}{-1.33} & \cellcolor{gray!15}{3.73} & \cellcolor{gray!15}{0.52}\\
4-3 & 0.2 & -2.33 & 2.73 & 1.00\\*
\end{longtable}

De forma similar, o teste não aponta diferença significativa entre
grupos de tratamento, mas sim entre quase todos os blocos, considerando
\(\alpha = 0,05\). São as exceções de significância: bloco 1 com blocos
3 e 4 e bloco 5 com bloco 2.

\hypertarget{conferencia-manual-das-contas}{%
\subsection{Conferencia manual das
contas}\label{conferencia-manual-das-contas}}

\begin{verbatim}
## [1] 0.1211445
\end{verbatim}

\begin{verbatim}
## [1] 2.059181e-05
\end{verbatim}

\begin{longtable}{cccccc}
\toprule
fontes & gl & sumsq & meansq & Festat & pvalor\\
\midrule
\endfirsthead
\multicolumn{6}{@{}l}{\textit{(continued)}}\\
\toprule
fontes & gl & sumsq & meansq & Festat & pvalor\\
\midrule
\endhead

\endfoot
\bottomrule
\endlastfoot
\cellcolor{gray!15}{bolts} & \cellcolor{gray!15}{4} & \cellcolor{gray!15}{157.00} & \cellcolor{gray!15}{39.25} & \cellcolor{gray!15}{21.606} & \cellcolor{gray!15}{0.1211}\\
chemicals & 3 & 12.95 & 4.32 & 2.376 & 0.0000\\
\cellcolor{gray!15}{residuals} & \cellcolor{gray!15}{12} & \cellcolor{gray!15}{21.80} & \cellcolor{gray!15}{1.82} & \cellcolor{gray!15}{NA} & \cellcolor{gray!15}{NA}\\*
\end{longtable}

Calcula-se inicialmente o valor crítico para a distância entre as médias
das amostras, considerando a distribuição de Tukey:

\begin{align}
    T_\alpha &= q_\alpha(a, f) \sqrt{\frac{MS_E}{n}}\\
    &= 4.19 \cdot \sqrt{\frac{MS_E}{4}}\\
    &= 2.829
\end{align}

\begin{longtable}{cccc}
\toprule
comparacao & distancias & n & pvalor\\
\midrule
\endfirsthead
\multicolumn{4}{@{}l}{\textit{(continued)}}\\
\toprule
comparacao & distancias & n & pvalor\\
\midrule
\endhead

\endfoot
\bottomrule
\endlastfoot
\cellcolor{gray!15}{media trat 12} & \cellcolor{gray!15}{0.80} & \cellcolor{gray!15}{4} & \cellcolor{gray!15}{0.8348}\\
media trat 13 & 1.80 & 4 & 0.2829\\
\cellcolor{gray!15}{media trat 14} & \cellcolor{gray!15}{2.00} & \cellcolor{gray!15}{4} & \cellcolor{gray!15}{0.2083}\\
media trat 23 & 1.00 & 4 & 0.7250\\
\cellcolor{gray!15}{media trat 24} & \cellcolor{gray!15}{1.20} & \cellcolor{gray!15}{4} & \cellcolor{gray!15}{0.6038}\\
media trat 34 & 0.20 & 4 & 0.9966\\
\cellcolor{gray!15}{media bloco 12} & \cellcolor{gray!15}{5.00} & \cellcolor{gray!15}{5} & \cellcolor{gray!15}{0.0016}\\
media bloco 13 & 2.00 & 5 & 0.2814\\
\cellcolor{gray!15}{media bloco 14} & \cellcolor{gray!15}{0.75} & \cellcolor{gray!15}{5} & \cellcolor{gray!15}{0.9296}\\
media bloco 15 & 5.00 & 5 & 0.0016\\
\cellcolor{gray!15}{media bloco 23} & \cellcolor{gray!15}{7.00} & \cellcolor{gray!15}{5} & \cellcolor{gray!15}{0.0001}\\
media bloco 24 & 4.25 & 5 & 0.0057\\
\cellcolor{gray!15}{media bloco 25} & \cellcolor{gray!15}{0.00} & \cellcolor{gray!15}{5} & \cellcolor{gray!15}{1.0000}\\
media bloco 34 & 2.75 & 5 & 0.0831\\
\cellcolor{gray!15}{media bloco 35} & \cellcolor{gray!15}{7.00} & \cellcolor{gray!15}{5} & \cellcolor{gray!15}{0.0001}\\
media bloco 45 & 4.25 & 5 & 0.0057\\*
\end{longtable}

\hypertarget{normalidade-e-homocedasticidade}{%
\subsection{Normalidade e
homocedasticidade}\label{normalidade-e-homocedasticidade}}

Conforme ja testado sobre os resíduos do modelo de análise variância, os
dados de fato não cumprem o pressuposto de normalidade. No entanto, são
homocedásticos.

\begin{longtable}{ccccc}
\toprule
statistic & p.value & df & df.residual & fonte\\
\midrule
\endfirsthead
\multicolumn{5}{@{}l}{\textit{(continued)}}\\
\toprule
statistic & p.value & df & df.residual & fonte\\
\midrule
\endhead

\endfoot
\bottomrule
\endlastfoot
\cellcolor{gray!15}{0.5815} & \cellcolor{gray!15}{0.6357} & \cellcolor{gray!15}{3} & \cellcolor{gray!15}{16} & \cellcolor{gray!15}{Chem - trat}\\
0.2400 & 0.9113 & 4 & 15 & Bolts - bloco\\*
\end{longtable}

Por último, verifica-se possível aditividade de efeito de tratamento com
efeito de bloco. Conforme o p-valor obtido a seguir para o teste de
aditividade, cuja hipótese nula é a aditividade completa do modelo (isto
é, não há um \(\lambda_{ij}\) diferente de zero, o qual representaria
interação de efeitos), pode-se considerar que o modelo é completamente
aditivo.

\begin{verbatim}
## [1] 0.7396448
\end{verbatim}

\begin{verbatim}
## [1] 0.7508062
\end{verbatim}

Como foi rejeitada a hipótese de normalidade, realiza-se o teste de
Friendman, não paramétrico, para avaliação da diferença entre os
tratamentos.

\begin{longtable}{cccc}
\toprule
chi-squared & p-value & df & method\\
\midrule
\endfirsthead
\multicolumn{4}{@{}l}{\textit{(continued)}}\\
\toprule
chi-squared & p-value & df & method\\
\midrule
\endhead

\endfoot
\bottomrule
\endlastfoot
\cellcolor{gray!15}{6} & \cellcolor{gray!15}{0.1116} & \cellcolor{gray!15}{3} & \cellcolor{gray!15}{Friedman rank sum test}\\*
\end{longtable}

\end{document}
