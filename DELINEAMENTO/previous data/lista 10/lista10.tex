% Options for packages loaded elsewhere
\PassOptionsToPackage{unicode}{hyperref}
\PassOptionsToPackage{hyphens}{url}
%
\documentclass[
]{article}
\usepackage{amsmath,amssymb}
\usepackage{lmodern}
\usepackage{iftex}
\ifPDFTeX
  \usepackage[T1]{fontenc}
  \usepackage[utf8]{inputenc}
  \usepackage{textcomp} % provide euro and other symbols
\else % if luatex or xetex
  \usepackage{unicode-math}
  \defaultfontfeatures{Scale=MatchLowercase}
  \defaultfontfeatures[\rmfamily]{Ligatures=TeX,Scale=1}
\fi
% Use upquote if available, for straight quotes in verbatim environments
\IfFileExists{upquote.sty}{\usepackage{upquote}}{}
\IfFileExists{microtype.sty}{% use microtype if available
  \usepackage[]{microtype}
  \UseMicrotypeSet[protrusion]{basicmath} % disable protrusion for tt fonts
}{}
\makeatletter
\@ifundefined{KOMAClassName}{% if non-KOMA class
  \IfFileExists{parskip.sty}{%
    \usepackage{parskip}
  }{% else
    \setlength{\parindent}{0pt}
    \setlength{\parskip}{6pt plus 2pt minus 1pt}}
}{% if KOMA class
  \KOMAoptions{parskip=half}}
\makeatother
\usepackage{xcolor}
\usepackage[margin=1in]{geometry}
\usepackage{graphicx}
\makeatletter
\def\maxwidth{\ifdim\Gin@nat@width>\linewidth\linewidth\else\Gin@nat@width\fi}
\def\maxheight{\ifdim\Gin@nat@height>\textheight\textheight\else\Gin@nat@height\fi}
\makeatother
% Scale images if necessary, so that they will not overflow the page
% margins by default, and it is still possible to overwrite the defaults
% using explicit options in \includegraphics[width, height, ...]{}
\setkeys{Gin}{width=\maxwidth,height=\maxheight,keepaspectratio}
% Set default figure placement to htbp
\makeatletter
\def\fps@figure{htbp}
\makeatother
\setlength{\emergencystretch}{3em} % prevent overfull lines
\providecommand{\tightlist}{%
  \setlength{\itemsep}{0pt}\setlength{\parskip}{0pt}}
\setcounter{secnumdepth}{-\maxdimen} % remove section numbering
\usepackage{helvet} \renewcommand\familydefault{\sfdefault}
\usepackage{booktabs}
\usepackage{longtable}
\usepackage{array}
\usepackage{multirow}
\usepackage{wrapfig}
\usepackage{float}
\usepackage{colortbl}
\usepackage{pdflscape}
\usepackage{tabu}
\usepackage{threeparttable}
\usepackage{threeparttablex}
\usepackage[normalem]{ulem}
\usepackage{makecell}
\usepackage{xcolor}
\ifLuaTeX
  \usepackage{selnolig}  % disable illegal ligatures
\fi
\IfFileExists{bookmark.sty}{\usepackage{bookmark}}{\usepackage{hyperref}}
\IfFileExists{xurl.sty}{\usepackage{xurl}}{} % add URL line breaks if available
\urlstyle{same} % disable monospaced font for URLs
\hypersetup{
  pdftitle={Lista 2 modulo 3},
  pdfauthor={César A. Galvão - 19/0011572},
  hidelinks,
  pdfcreator={LaTeX via pandoc}}

\title{Lista 2 modulo 3}
\author{César A. Galvão - 19/0011572}
\date{2022-09-20}

\begin{document}
\maketitle

\newpage{}

{
\setcounter{tocdepth}{3}
\tableofcontents
}
\let\oldsection\section
\renewcommand\section{\clearpage\oldsection}

\hypertarget{section}{%
\section*{}\label{section}}
\addcontentsline{toc}{section}{}

\hypertarget{anuxe1lise-de-variuxe2ncia}{%
\subsection{Análise de Variância}\label{anuxe1lise-de-variuxe2ncia}}

A tabela de análise de variância a seguir sugere como significativos
apenas para os efeitos principais dos fatores, desconsiderando a
interação entre eles.

\begin{longtable}{cccccc}
\toprule
term & df & sumsq & meansq & statistic & p.value\\
\midrule
\endfirsthead
\multicolumn{6}{@{}l}{\textit{(continued)}}\\
\toprule
term & df & sumsq & meansq & statistic & p.value\\
\midrule
\endhead

\endfoot
\bottomrule
\endlastfoot
\cellcolor{gray!15}{as.factor(fa)} & \cellcolor{gray!15}{1} & \cellcolor{gray!15}{36.00} & \cellcolor{gray!15}{36.000} & \cellcolor{gray!15}{57.6} & \cellcolor{gray!15}{0.0001}\\
as.factor(fb) & 1 & 20.25 & 20.250 & 32.4 & 0.0005\\
\cellcolor{gray!15}{as.factor(fc)} & \cellcolor{gray!15}{1} & \cellcolor{gray!15}{12.25} & \cellcolor{gray!15}{12.250} & \cellcolor{gray!15}{19.6} & \cellcolor{gray!15}{0.0022}\\
as.factor(fa):as.factor(fb) & 1 & 2.25 & 2.250 & 3.6 & 0.0943\\
\cellcolor{gray!15}{as.factor(fa):as.factor(fc)} & \cellcolor{gray!15}{1} & \cellcolor{gray!15}{0.25} & \cellcolor{gray!15}{0.250} & \cellcolor{gray!15}{0.4} & \cellcolor{gray!15}{0.5447}\\
as.factor(fb):as.factor(fc) & 1 & 1.00 & 1.000 & 1.6 & 0.2415\\
\cellcolor{gray!15}{as.factor(fa):as.factor(fb):as.factor(fc)} & \cellcolor{gray!15}{1} & \cellcolor{gray!15}{1.00} & \cellcolor{gray!15}{1.000} & \cellcolor{gray!15}{1.6} & \cellcolor{gray!15}{0.2415}\\
Residuals & 8 & 5.00 & 0.625 & NA & NA\\*
\end{longtable}

Considera-se como modelo, portanto, apenas a análise de variância
modelada com os fatores principais:

\begin{longtable}{cccccc}
\toprule
term & df & sumsq & meansq & statistic & p.value\\
\midrule
\endfirsthead
\multicolumn{6}{@{}l}{\textit{(continued)}}\\
\toprule
term & df & sumsq & meansq & statistic & p.value\\
\midrule
\endhead

\endfoot
\bottomrule
\endlastfoot
\cellcolor{gray!15}{as.factor(fa)} & \cellcolor{gray!15}{1} & \cellcolor{gray!15}{36.00} & \cellcolor{gray!15}{36.0000} & \cellcolor{gray!15}{45.4737} & \cellcolor{gray!15}{0.0000}\\
as.factor(fb) & 1 & 20.25 & 20.2500 & 25.5789 & 0.0003\\
\cellcolor{gray!15}{as.factor(fc)} & \cellcolor{gray!15}{1} & \cellcolor{gray!15}{12.25} & \cellcolor{gray!15}{12.2500} & \cellcolor{gray!15}{15.4737} & \cellcolor{gray!15}{0.0020}\\
Residuals & 12 & 9.50 & 0.7917 & NA & NA\\*
\end{longtable}

\hypertarget{modelo-de-regressuxe3o-linear}{%
\subsection{Modelo de regressão
linear}\label{modelo-de-regressuxe3o-linear}}

O modelo de regressão ajustado é:

\begin{align}
\hat{y} &= \hat{\beta}_0 + \hat{\beta}_1X_1 + \hat{\beta}_2X_2 + \hat{\beta}_3X_3 \nonumber \\
&= 1 + 1.5X_1 + 1.125X_2 + 0.875X_3
\end{align}

\hypertarget{anuxe1lise-dos-resuxedduos}{%
\subsection{Análise dos resíduos}\label{anuxe1lise-dos-resuxedduos}}

Testa-se normalidade dos resíduos utilizando o teste de Shapiro-Wilk. É
possível observar na tabela abaixo que o teste não rejeita normalidade
dos resíduos do modelo.

\begin{longtable}{ccc}
\toprule
statistic & p.value & method\\
\midrule
\endfirsthead
\multicolumn{3}{@{}l}{\textit{(continued)}}\\
\toprule
statistic & p.value & method\\
\midrule
\endhead

\endfoot
\bottomrule
\endlastfoot
\cellcolor{gray!15}{0.9270108} & \cellcolor{gray!15}{0.2185152} & \cellcolor{gray!15}{Shapiro-Wilk normality test}\\*
\end{longtable}

Além disso, o teste Breusch-Pagan para homocedasticidade não rejeita
homocedasticidade dos resíduos.

\begin{longtable}{cccc}
\toprule
statistic & p.value & parameter & method\\
\midrule
\endfirsthead
\multicolumn{4}{@{}l}{\textit{(continued)}}\\
\toprule
statistic & p.value & parameter & method\\
\midrule
\endhead

\endfoot
\bottomrule
\endlastfoot
\cellcolor{gray!15}{0.3629} & \cellcolor{gray!15}{0.9478} & \cellcolor{gray!15}{3} & \cellcolor{gray!15}{studentized Breusch-Pagan test}\\*
\end{longtable}

Conclui-se pela adequação dos modelos de regressão e anova adotados.

\hypertarget{curvatura}{%
\subsection{Curvatura}\label{curvatura}}

A média dos pontos centrais é 0.955 e a média dos pontos fatoriais é 1,
fato que indica que os pontos estão no mesmo plano, já que são valores
próximos.

\begin{verbatim}
## [1] 0.03058837
\end{verbatim}

No entanto, o p-valor obtido para o teste de curatura indica que há sim
uma curvatura.

\end{document}
