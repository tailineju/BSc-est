% Options for packages loaded elsewhere
\PassOptionsToPackage{unicode}{hyperref}
\PassOptionsToPackage{hyphens}{url}
%
\documentclass[
]{article}
\usepackage{amsmath,amssymb}
\usepackage{lmodern}
\usepackage{iftex}
\ifPDFTeX
  \usepackage[T1]{fontenc}
  \usepackage[utf8]{inputenc}
  \usepackage{textcomp} % provide euro and other symbols
\else % if luatex or xetex
  \usepackage{unicode-math}
  \defaultfontfeatures{Scale=MatchLowercase}
  \defaultfontfeatures[\rmfamily]{Ligatures=TeX,Scale=1}
\fi
% Use upquote if available, for straight quotes in verbatim environments
\IfFileExists{upquote.sty}{\usepackage{upquote}}{}
\IfFileExists{microtype.sty}{% use microtype if available
  \usepackage[]{microtype}
  \UseMicrotypeSet[protrusion]{basicmath} % disable protrusion for tt fonts
}{}
\makeatletter
\@ifundefined{KOMAClassName}{% if non-KOMA class
  \IfFileExists{parskip.sty}{%
    \usepackage{parskip}
  }{% else
    \setlength{\parindent}{0pt}
    \setlength{\parskip}{6pt plus 2pt minus 1pt}}
}{% if KOMA class
  \KOMAoptions{parskip=half}}
\makeatother
\usepackage{xcolor}
\usepackage[margin=1in]{geometry}
\usepackage{graphicx}
\makeatletter
\def\maxwidth{\ifdim\Gin@nat@width>\linewidth\linewidth\else\Gin@nat@width\fi}
\def\maxheight{\ifdim\Gin@nat@height>\textheight\textheight\else\Gin@nat@height\fi}
\makeatother
% Scale images if necessary, so that they will not overflow the page
% margins by default, and it is still possible to overwrite the defaults
% using explicit options in \includegraphics[width, height, ...]{}
\setkeys{Gin}{width=\maxwidth,height=\maxheight,keepaspectratio}
% Set default figure placement to htbp
\makeatletter
\def\fps@figure{htbp}
\makeatother
\setlength{\emergencystretch}{3em} % prevent overfull lines
\providecommand{\tightlist}{%
  \setlength{\itemsep}{0pt}\setlength{\parskip}{0pt}}
\setcounter{secnumdepth}{5}
\usepackage{helvet} \renewcommand\familydefault{\sfdefault}
\usepackage{booktabs}
\usepackage{longtable}
\usepackage{array}
\usepackage{multirow}
\usepackage{wrapfig}
\usepackage{float}
\usepackage{colortbl}
\usepackage{pdflscape}
\usepackage{tabu}
\usepackage{threeparttable}
\usepackage{threeparttablex}
\usepackage[normalem]{ulem}
\usepackage{makecell}
\usepackage{xcolor}
\ifLuaTeX
  \usepackage{selnolig}  % disable illegal ligatures
\fi
\IfFileExists{bookmark.sty}{\usepackage{bookmark}}{\usepackage{hyperref}}
\IfFileExists{xurl.sty}{\usepackage{xurl}}{} % add URL line breaks if available
\urlstyle{same} % disable monospaced font for URLs
\hypersetup{
  pdftitle={Exercício de laboratorio 11},
  pdfauthor={César Galvão - 19/0011572},
  hidelinks,
  pdfcreator={LaTeX via pandoc}}

\title{Exercício de laboratorio 11}
\author{César Galvão - 19/0011572}
\date{2022-09-21}

\begin{document}
\maketitle

\newpage{}

{
\setcounter{tocdepth}{3}
\tableofcontents
}
\let\oldsection\section
\renewcommand\section{\clearpage\oldsection}

\hypertarget{questuxe3o-1}{%
\section{Questão 1}\label{questuxe3o-1}}

\hypertarget{entrada-de-dados}{%
\subsection{Entrada de dados}\label{entrada-de-dados}}

\hypertarget{curvatura}{%
\subsection{Curvatura}\label{curvatura}}

\begin{longtable}{cc}
\toprule
$\bar{X}_{Fatorial}$ & $\bar{X}_{central}$\\
\midrule
\endfirsthead
\multicolumn{2}{@{}l}{\textit{(continued)}}\\
\toprule
$\bar{X}_{Fatorial}$ & $\bar{X}_{central}$\\
\midrule
\endhead

\endfoot
\bottomrule
\endlastfoot
\cellcolor{gray!15}{84} & \cellcolor{gray!15}{84.2}\\*
\end{longtable}

Pelo valor observado nas médias dos pontos fatoriais e dos pontos
centrais, há um indicativo de que não há curvatura, já que os valores
são muito próximos.

O teste F resulta em p-valor igual a 0.3534. Dessa forma há evidências
para concluir que não há curvatura.

Analogamente, o teste de ANOVA sugere que há apenas efeito da
temperatura na resposta obtida.

\begin{longtable}{cccccc}
\toprule
term & df & sumsq & meansq & statistic & p.value\\
\midrule
\endfirsthead
\multicolumn{6}{@{}l}{\textit{(continued)}}\\
\toprule
term & df & sumsq & meansq & statistic & p.value\\
\midrule
\endhead

\endfoot
\bottomrule
\endlastfoot
\cellcolor{gray!15}{as.factor(aa11)} & \cellcolor{gray!15}{2} & \cellcolor{gray!15}{0.33} & \cellcolor{gray!15}{0.1650} & \cellcolor{gray!15}{2.475} & \cellcolor{gray!15}{0.2318}\\
as.factor(bb11) & 1 & 2.89 & 2.8900 & 43.350 & 0.0071\\
\cellcolor{gray!15}{as.factor(aa11):as.factor(bb11)} & \cellcolor{gray!15}{1} & \cellcolor{gray!15}{0.04} & \cellcolor{gray!15}{0.0400} & \cellcolor{gray!15}{0.600} & \cellcolor{gray!15}{0.4950}\\
Residuals & 3 & 0.20 & 0.0667 & NA & NA\\*
\end{longtable}

O modelo de regressão corrobora o resultado anterior de que apenas a
temperatura tem efeito na resposta do experimento.

\begin{longtable}{ccccc}
\toprule
term & estimate & std.error & statistic & p.value\\
\midrule
\endfirsthead
\multicolumn{5}{@{}l}{\textit{(continued)}}\\
\toprule
term & estimate & std.error & statistic & p.value\\
\midrule
\endhead

\endfoot
\bottomrule
\endlastfoot
\cellcolor{gray!15}{(Intercept)} & \cellcolor{gray!15}{84.10} & \cellcolor{gray!15}{0.0935} & \cellcolor{gray!15}{899.0668} & \cellcolor{gray!15}{0.0000}\\
aa11 & 0.25 & 0.1323 & 1.8898 & 0.1318\\
\cellcolor{gray!15}{bb11} & \cellcolor{gray!15}{0.85} & \cellcolor{gray!15}{0.1323} & \cellcolor{gray!15}{6.4254} & \cellcolor{gray!15}{0.0030}\\
aa11:bb11 & -0.10 & 0.1323 & -0.7559 & 0.4918\\*
\end{longtable}

Com base neste resultado, o modelo de regressão será ajustado para:

\begin{align}
\hat{y} = \beta_0 + \beta_1X_1
\end{align}

onde:

\begin{align}
\hat{y} = 84.10 - 0.85X_1
\end{align}

\hypertarget{caminho-de-maior-subida}{%
\subsection{Caminho de maior subida}\label{caminho-de-maior-subida}}

\hypertarget{questuxe3o-2}{%
\section{Questão 2}\label{questuxe3o-2}}

\hypertarget{entrada-de-dados-1}{%
\subsection{Entrada de dados}\label{entrada-de-dados-1}}

\begin{verbatim}
##                                                                                  Df
## as.factor(laserpower)                                                             1
## as.factor(cellsize)                                                               1
## as.factor(pulsefreq)                                                              1
## as.factor(wspeed)                                                                 1
## as.factor(laserpower):as.factor(cellsize)                                         1
## as.factor(laserpower):as.factor(pulsefreq)                                        1
## as.factor(cellsize):as.factor(pulsefreq)                                          1
## as.factor(laserpower):as.factor(wspeed)                                           1
## as.factor(cellsize):as.factor(wspeed)                                             1
## as.factor(pulsefreq):as.factor(wspeed)                                            1
## as.factor(laserpower):as.factor(cellsize):as.factor(pulsefreq)                    1
## as.factor(laserpower):as.factor(cellsize):as.factor(wspeed)                       1
## as.factor(laserpower):as.factor(pulsefreq):as.factor(wspeed)                      1
## as.factor(cellsize):as.factor(pulsefreq):as.factor(wspeed)                        1
## as.factor(laserpower):as.factor(cellsize):as.factor(pulsefreq):as.factor(wspeed)  1
##                                                                                   Sum Sq
## as.factor(laserpower)                                                            0.10240
## as.factor(cellsize)                                                              0.07022
## as.factor(pulsefreq)                                                             0.00160
## as.factor(wspeed)                                                                0.05063
## as.factor(laserpower):as.factor(cellsize)                                        0.01210
## as.factor(laserpower):as.factor(pulsefreq)                                       0.00123
## as.factor(cellsize):as.factor(pulsefreq)                                         0.00250
## as.factor(laserpower):as.factor(wspeed)                                          0.00040
## as.factor(cellsize):as.factor(wspeed)                                            0.00562
## as.factor(pulsefreq):as.factor(wspeed)                                           0.00040
## as.factor(laserpower):as.factor(cellsize):as.factor(pulsefreq)                   0.00203
## as.factor(laserpower):as.factor(cellsize):as.factor(wspeed)                      0.00160
## as.factor(laserpower):as.factor(pulsefreq):as.factor(wspeed)                     0.00123
## as.factor(cellsize):as.factor(pulsefreq):as.factor(wspeed)                       0.00160
## as.factor(laserpower):as.factor(cellsize):as.factor(pulsefreq):as.factor(wspeed) 0.00002
##                                                                                  Mean Sq
## as.factor(laserpower)                                                            0.10240
## as.factor(cellsize)                                                              0.07022
## as.factor(pulsefreq)                                                             0.00160
## as.factor(wspeed)                                                                0.05063
## as.factor(laserpower):as.factor(cellsize)                                        0.01210
## as.factor(laserpower):as.factor(pulsefreq)                                       0.00123
## as.factor(cellsize):as.factor(pulsefreq)                                         0.00250
## as.factor(laserpower):as.factor(wspeed)                                          0.00040
## as.factor(cellsize):as.factor(wspeed)                                            0.00562
## as.factor(pulsefreq):as.factor(wspeed)                                           0.00040
## as.factor(laserpower):as.factor(cellsize):as.factor(pulsefreq)                   0.00203
## as.factor(laserpower):as.factor(cellsize):as.factor(wspeed)                      0.00160
## as.factor(laserpower):as.factor(pulsefreq):as.factor(wspeed)                     0.00123
## as.factor(cellsize):as.factor(pulsefreq):as.factor(wspeed)                       0.00160
## as.factor(laserpower):as.factor(cellsize):as.factor(pulsefreq):as.factor(wspeed) 0.00002
\end{verbatim}

\begin{longtable}{cccc}
\toprule
term & df & sumsq & meansq\\
\midrule
\endfirsthead
\multicolumn{4}{@{}l}{\textit{(continued)}}\\
\toprule
term & df & sumsq & meansq\\
\midrule
\endhead

\endfoot
\bottomrule
\endlastfoot
\cellcolor{gray!15}{as.factor(laserpower)} & \cellcolor{gray!15}{1} & \cellcolor{gray!15}{0.1024} & \cellcolor{gray!15}{0.1024}\\
as.factor(cellsize) & 1 & 0.0702 & 0.0702\\
\cellcolor{gray!15}{as.factor(pulsefreq)} & \cellcolor{gray!15}{1} & \cellcolor{gray!15}{0.0016} & \cellcolor{gray!15}{0.0016}\\
as.factor(wspeed) & 1 & 0.0506 & 0.0506\\
\cellcolor{gray!15}{as.factor(laserpower):as.factor(cellsize)} & \cellcolor{gray!15}{1} & \cellcolor{gray!15}{0.0121} & \cellcolor{gray!15}{0.0121}\\
as.factor(laserpower):as.factor(pulsefreq) & 1 & 0.0012 & 0.0012\\
\cellcolor{gray!15}{as.factor(cellsize):as.factor(pulsefreq)} & \cellcolor{gray!15}{1} & \cellcolor{gray!15}{0.0025} & \cellcolor{gray!15}{0.0025}\\
as.factor(laserpower):as.factor(wspeed) & 1 & 0.0004 & 0.0004\\
\cellcolor{gray!15}{as.factor(cellsize):as.factor(wspeed)} & \cellcolor{gray!15}{1} & \cellcolor{gray!15}{0.0056} & \cellcolor{gray!15}{0.0056}\\
as.factor(pulsefreq):as.factor(wspeed) & 1 & 0.0004 & 0.0004\\
\cellcolor{gray!15}{as.factor(laserpower):as.factor(cellsize):as.factor(pulsefreq)} & \cellcolor{gray!15}{1} & \cellcolor{gray!15}{0.0020} & \cellcolor{gray!15}{0.0020}\\
as.factor(laserpower):as.factor(cellsize):as.factor(wspeed) & 1 & 0.0016 & 0.0016\\
\cellcolor{gray!15}{as.factor(laserpower):as.factor(pulsefreq):as.factor(wspeed)} & \cellcolor{gray!15}{1} & \cellcolor{gray!15}{0.0012} & \cellcolor{gray!15}{0.0012}\\
as.factor(cellsize):as.factor(pulsefreq):as.factor(wspeed) & 1 & 0.0016 & 0.0016\\
\cellcolor{gray!15}{as.factor(laserpower):as.factor(cellsize):as.factor(pulsefreq):as.factor(wspeed)} & \cellcolor{gray!15}{1} & \cellcolor{gray!15}{0.0000} & \cellcolor{gray!15}{0.0000}\\*
\end{longtable}

Como o experimento tem apenas uma observação para casa combinação de
fatores, não há como calcular a variabilidade dentro de cada tratamento.
Desse modo, a contribuição de cada fator é avaliada pela contribuição
que cada soma de quadrados de cada fator tem na soma de quadrados total.
Pela ANOVA inicial, nota-se que os fatores que contribuem mais para a
soma de quadrados total são Laser Power, Cell Size e Writing Speed, bem
como a interação entre Laser Power e Cell Size. dessa forma, o modelo
foi ajustado levando em consideração esses fatores.

\begin{verbatim}
##                                           Df  Sum Sq Mean Sq F value   Pr(>F)
## as.factor(laserpower)                      1 0.10240 0.10240  61.594 1.39e-05
## as.factor(cellsize)                        1 0.07022 0.07022  42.241 6.90e-05
## as.factor(pulsefreq)                       1 0.00160 0.00160   0.962 0.349719
## as.factor(wspeed)                          1 0.05063 0.05063  30.451 0.000255
## as.factor(laserpower):as.factor(cellsize)  1 0.01210 0.01210   7.278 0.022397
## Residuals                                 10 0.01663 0.00166                 
##                                              
## as.factor(laserpower)                     ***
## as.factor(cellsize)                       ***
## as.factor(pulsefreq)                         
## as.factor(wspeed)                         ***
## as.factor(laserpower):as.factor(cellsize) *  
## Residuals                                    
## ---
## Signif. codes:  0 '***' 0.001 '**' 0.01 '*' 0.05 '.' 0.1 ' ' 1
\end{verbatim}

\begin{longtable}{cccccc}
\toprule
term & df & sumsq & meansq & statistic & p.value\\
\midrule
\endfirsthead
\multicolumn{6}{@{}l}{\textit{(continued)}}\\
\toprule
term & df & sumsq & meansq & statistic & p.value\\
\midrule
\endhead

\endfoot
\bottomrule
\endlastfoot
\cellcolor{gray!15}{as.factor(laserpower)} & \cellcolor{gray!15}{1} & \cellcolor{gray!15}{0.1024} & \cellcolor{gray!15}{0.1024} & \cellcolor{gray!15}{61.5940} & \cellcolor{gray!15}{0.0000}\\
as.factor(cellsize) & 1 & 0.0702 & 0.0702 & 42.2406 & 0.0001\\
\cellcolor{gray!15}{as.factor(pulsefreq)} & \cellcolor{gray!15}{1} & \cellcolor{gray!15}{0.0016} & \cellcolor{gray!15}{0.0016} & \cellcolor{gray!15}{0.9624} & \cellcolor{gray!15}{0.3497}\\
as.factor(wspeed) & 1 & 0.0506 & 0.0506 & 30.4511 & 0.0003\\
\cellcolor{gray!15}{as.factor(laserpower):as.factor(cellsize)} & \cellcolor{gray!15}{1} & \cellcolor{gray!15}{0.0121} & \cellcolor{gray!15}{0.0121} & \cellcolor{gray!15}{7.2782} & \cellcolor{gray!15}{0.0224}\\
Residuals & 10 & 0.0166 & 0.0017 & NA & NA\\*
\end{longtable}

Com base na nova análise de variancia, pode-se observar que o fator
Pulse Frequncy não afeta o resultado do experimento.

Assim, ajustamos o modelo de regressão para:

\begin{longtable}{ccccc}
\toprule
term & estimate & std.error & statistic & p.value\\
\midrule
\endfirsthead
\multicolumn{5}{@{}l}{\textit{(continued)}}\\
\toprule
term & estimate & std.error & statistic & p.value\\
\midrule
\endhead

\endfoot
\bottomrule
\endlastfoot
\cellcolor{gray!15}{(Intercept)} & \cellcolor{gray!15}{0.7313} & \cellcolor{gray!15}{0.0228} & \cellcolor{gray!15}{32.1369} & \cellcolor{gray!15}{0.0000}\\
as.factor(laserpower)1 & 0.2150 & 0.0288 & 7.4699 & 0.0000\\
\cellcolor{gray!15}{as.factor(cellsize)1} & \cellcolor{gray!15}{-0.0775} & \cellcolor{gray!15}{0.0288} & \cellcolor{gray!15}{-2.6926} & \cellcolor{gray!15}{0.0209}\\
as.factor(wspeed)1 & -0.1125 & 0.0204 & -5.5277 & 0.0002\\
\cellcolor{gray!15}{as.factor(laserpower)1:as.factor(cellsize)1} & \cellcolor{gray!15}{-0.1100} & \cellcolor{gray!15}{0.0407} & \cellcolor{gray!15}{-2.7024} & \cellcolor{gray!15}{0.0206}\\*
\end{longtable}

\begin{align}
\hat{y} = \hat{\beta}_0 + \hat{\beta}_1X_1 + \hat{\beta}_2X_2 + \hat{\beta}_3X_3 + \hat{\beta}_4X_1X_2 
\end{align}

onde:

\begin{align}
\hat{y} = 0.73125 + 0.215X_1 - 0.0775X_2 - 0.1125X_3 - 0.11X_1X_2
\end{align}

\hypertarget{anuxe1lise-de-resuxedduos-do-modelo-final}{%
\subsection{Análise de resíduos do modelo
final}\label{anuxe1lise-de-resuxedduos-do-modelo-final}}

\begin{longtable}{ccc}
\toprule
statistic & p.value & method\\
\midrule
\endfirsthead
\multicolumn{3}{@{}l}{\textit{(continued)}}\\
\toprule
statistic & p.value & method\\
\midrule
\endhead

\endfoot
\bottomrule
\endlastfoot
\cellcolor{gray!15}{0.9278807} & \cellcolor{gray!15}{0.2256039} & \cellcolor{gray!15}{Shapiro-Wilk normality test}\\*
\end{longtable}

\begin{longtable}{cccc}
\toprule
statistic & p.value & parameter & method\\
\midrule
\endfirsthead
\multicolumn{4}{@{}l}{\textit{(continued)}}\\
\toprule
statistic & p.value & parameter & method\\
\midrule
\endhead

\endfoot
\bottomrule
\endlastfoot
\cellcolor{gray!15}{6.4083} & \cellcolor{gray!15}{0.1707} & \cellcolor{gray!15}{4} & \cellcolor{gray!15}{studentized Breusch-Pagan test}\\*
\end{longtable}

\end{document}
