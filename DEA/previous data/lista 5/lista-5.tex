% Options for packages loaded elsewhere
\PassOptionsToPackage{unicode}{hyperref}
\PassOptionsToPackage{hyphens}{url}
%
\documentclass[
]{article}
\usepackage{amsmath,amssymb}
\usepackage{lmodern}
\usepackage{iftex}
\ifPDFTeX
  \usepackage[T1]{fontenc}
  \usepackage[utf8]{inputenc}
  \usepackage{textcomp} % provide euro and other symbols
\else % if luatex or xetex
  \usepackage{unicode-math}
  \defaultfontfeatures{Scale=MatchLowercase}
  \defaultfontfeatures[\rmfamily]{Ligatures=TeX,Scale=1}
\fi
% Use upquote if available, for straight quotes in verbatim environments
\IfFileExists{upquote.sty}{\usepackage{upquote}}{}
\IfFileExists{microtype.sty}{% use microtype if available
  \usepackage[]{microtype}
  \UseMicrotypeSet[protrusion]{basicmath} % disable protrusion for tt fonts
}{}
\makeatletter
\@ifundefined{KOMAClassName}{% if non-KOMA class
  \IfFileExists{parskip.sty}{%
    \usepackage{parskip}
  }{% else
    \setlength{\parindent}{0pt}
    \setlength{\parskip}{6pt plus 2pt minus 1pt}}
}{% if KOMA class
  \KOMAoptions{parskip=half}}
\makeatother
\usepackage{xcolor}
\IfFileExists{xurl.sty}{\usepackage{xurl}}{} % add URL line breaks if available
\IfFileExists{bookmark.sty}{\usepackage{bookmark}}{\usepackage{hyperref}}
\hypersetup{
  pdftitle={Exercício de laboratório 5},
  pdfauthor={César A. Galvão - 19/0011572},
  hidelinks,
  pdfcreator={LaTeX via pandoc}}
\urlstyle{same} % disable monospaced font for URLs
\usepackage[margin=1in]{geometry}
\usepackage{graphicx}
\makeatletter
\def\maxwidth{\ifdim\Gin@nat@width>\linewidth\linewidth\else\Gin@nat@width\fi}
\def\maxheight{\ifdim\Gin@nat@height>\textheight\textheight\else\Gin@nat@height\fi}
\makeatother
% Scale images if necessary, so that they will not overflow the page
% margins by default, and it is still possible to overwrite the defaults
% using explicit options in \includegraphics[width, height, ...]{}
\setkeys{Gin}{width=\maxwidth,height=\maxheight,keepaspectratio}
% Set default figure placement to htbp
\makeatletter
\def\fps@figure{htbp}
\makeatother
\setlength{\emergencystretch}{3em} % prevent overfull lines
\providecommand{\tightlist}{%
  \setlength{\itemsep}{0pt}\setlength{\parskip}{0pt}}
\setcounter{secnumdepth}{-\maxdimen} % remove section numbering
\usepackage{helvet} \renewcommand\familydefault{\sfdefault}
\usepackage{booktabs}
\usepackage{longtable}
\usepackage{array}
\usepackage{multirow}
\usepackage{wrapfig}
\usepackage{float}
\usepackage{colortbl}
\usepackage{pdflscape}
\usepackage{tabu}
\usepackage{threeparttable}
\usepackage{threeparttablex}
\usepackage[normalem]{ulem}
\usepackage{makecell}
\usepackage{xcolor}
\ifLuaTeX
  \usepackage{selnolig}  % disable illegal ligatures
\fi

\title{Exercício de laboratório 5}
\usepackage{etoolbox}
\makeatletter
\providecommand{\subtitle}[1]{% add subtitle to \maketitle
  \apptocmd{\@title}{\par {\large #1 \par}}{}{}
}
\makeatother
\subtitle{Simulações Erro Tipo 2}
\author{César A. Galvão - 19/0011572}
\date{2022-08-12}

\begin{document}
\maketitle

\newpage{}

{
\setcounter{tocdepth}{2}
\tableofcontents
}
\let\oldsection\section
\renewcommand\section{\clearpage\oldsection}

\hypertarget{cenuxe1rio-1---tratamentos-significativos-e-blocos-nuxe3o-significativos}{%
\section{Cenário 1 - Tratamentos significativos e blocos não
significativos}\label{cenuxe1rio-1---tratamentos-significativos-e-blocos-nuxe3o-significativos}}

Consideramos dois modelos, o primeiro considerando apenas o efeito de
tratamentos e o segundo considerando efeito de blocos:

\begin{align}
  y_{ij} &= \mu + \tau_i + \varepsilon_{ij} \label{51}\\
  y_{ij} &= \mu + \tau_i + \beta_j + \varepsilon_{ij} \label{52}
\end{align}

Consideramos um cenário em que de fato há diferença entre quatro
tratamentos \((10, 20, -15, -15)\), porém os efeitos de blocos
constituem um vetor nulo \((0, 0, 0, 0)\). Considerando a variância dos
erros aleatórios como a mesma de
\(Var(y_{ij}) = Var(\varepsilon_{ij}) = \sigma^2 = 15^2\). Além disso,
considera-se \(\mu = 50\) e \(\alpha = 0,05\).

O erro tipo II é simulado gerando amostras aleatórias para ambos os
modelos e contando aqueles em que, pelo p-valor, não se rejeita a
hipótese nula.

Espera-se haver mais casos de erro tipo 2 quando a blocagem é
desnecessária, visto que se reduz os graus de liberdade do resíduo --
consequentemente aumentando o QMRES que é o denominador da estatística
de teste.

Para o cálculo analítico do erro tipo 2, utiliza-se o seguinte parâmetro
de não centralidade:

\begin{align}
  NCP = \phi^2 &= n \cdot \frac{\sum\limits_{i = 1}^{4} \tau_i^2}{\sigma^2}\\
  &= 4 \cdot \frac{3800}{225}
\end{align}

Além disso, considera-se \(an-a = 4\cdot 4 - 4 = 12\) graus de liberdade
para o modelo (\ref{51}) e \((a-1)(b-1) = 9\) graus de liberdade para o
modelo (\ref{52}).

Quando não se considera blocos, modelo (\ref{51}), obtem-se uma
proporção de 0.159 erros tipo 2. No modelo (\ref{52}), a proporção é de
0.205, corroborando a hipótese inicial. Os valores analíticos seguem a
mesma tendência -- 0.156 e 0.202 respectivamente.

\hypertarget{cenuxe1rio-2---tratamentos-significativos-e-blocos-significativos}{%
\section{Cenário 2 - Tratamentos significativos e blocos
significativos}\label{cenuxe1rio-2---tratamentos-significativos-e-blocos-significativos}}

É esperado que, quando a estrutura de bloco é necessária e não é
utilizada, mais variabilidade do sistema seja atribuída ao resíduo --
quando não seria devido ao erro aleatório. Isso ocorre devido a
inflacionamento da soma de quadrados do resíduo.

De fato, observa-se um erro tipo dois muito divergente daquele calculado
analiticamente e superior ao obtido empiricamente para a estrutura com
bloco.

\begin{longtable}{ccc}
\toprule
Modelo & Erro.empirico & Erro.teorico\\
\midrule
\endfirsthead
\multicolumn{3}{@{}l}{\textit{(continued)}}\\
\toprule
Modelo & Erro.empirico & Erro.teorico\\
\midrule
\endhead

\endfoot
\bottomrule
\endlastfoot
\cellcolor{gray!15}{sem bloco} & \cellcolor{gray!15}{0.689} & \cellcolor{gray!15}{0.156}\\
com bloco & 0.203 & 0.202\\*
\end{longtable}

\hypertarget{cenuxe1rio-3---tratamentos-nuxe3o-significativos-e-blocos-nuxe3o-significativos}{%
\section{Cenário 3 - Tratamentos não significativos e blocos não
significativos}\label{cenuxe1rio-3---tratamentos-nuxe3o-significativos-e-blocos-nuxe3o-significativos}}

\end{document}
