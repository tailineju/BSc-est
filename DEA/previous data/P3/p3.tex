% Options for packages loaded elsewhere
\PassOptionsToPackage{unicode}{hyperref}
\PassOptionsToPackage{hyphens}{url}
%
\documentclass[
]{article}
\usepackage{amsmath,amssymb}
\usepackage{lmodern}
\usepackage{iftex}
\ifPDFTeX
  \usepackage[T1]{fontenc}
  \usepackage[utf8]{inputenc}
  \usepackage{textcomp} % provide euro and other symbols
\else % if luatex or xetex
  \usepackage{unicode-math}
  \defaultfontfeatures{Scale=MatchLowercase}
  \defaultfontfeatures[\rmfamily]{Ligatures=TeX,Scale=1}
\fi
% Use upquote if available, for straight quotes in verbatim environments
\IfFileExists{upquote.sty}{\usepackage{upquote}}{}
\IfFileExists{microtype.sty}{% use microtype if available
  \usepackage[]{microtype}
  \UseMicrotypeSet[protrusion]{basicmath} % disable protrusion for tt fonts
}{}
\makeatletter
\@ifundefined{KOMAClassName}{% if non-KOMA class
  \IfFileExists{parskip.sty}{%
    \usepackage{parskip}
  }{% else
    \setlength{\parindent}{0pt}
    \setlength{\parskip}{6pt plus 2pt minus 1pt}}
}{% if KOMA class
  \KOMAoptions{parskip=half}}
\makeatother
\usepackage{xcolor}
\usepackage[margin=1in]{geometry}
\usepackage{color}
\usepackage{fancyvrb}
\newcommand{\VerbBar}{|}
\newcommand{\VERB}{\Verb[commandchars=\\\{\}]}
\DefineVerbatimEnvironment{Highlighting}{Verbatim}{commandchars=\\\{\}}
% Add ',fontsize=\small' for more characters per line
\usepackage{framed}
\definecolor{shadecolor}{RGB}{248,248,248}
\newenvironment{Shaded}{\begin{snugshade}}{\end{snugshade}}
\newcommand{\AlertTok}[1]{\textcolor[rgb]{0.94,0.16,0.16}{#1}}
\newcommand{\AnnotationTok}[1]{\textcolor[rgb]{0.56,0.35,0.01}{\textbf{\textit{#1}}}}
\newcommand{\AttributeTok}[1]{\textcolor[rgb]{0.77,0.63,0.00}{#1}}
\newcommand{\BaseNTok}[1]{\textcolor[rgb]{0.00,0.00,0.81}{#1}}
\newcommand{\BuiltInTok}[1]{#1}
\newcommand{\CharTok}[1]{\textcolor[rgb]{0.31,0.60,0.02}{#1}}
\newcommand{\CommentTok}[1]{\textcolor[rgb]{0.56,0.35,0.01}{\textit{#1}}}
\newcommand{\CommentVarTok}[1]{\textcolor[rgb]{0.56,0.35,0.01}{\textbf{\textit{#1}}}}
\newcommand{\ConstantTok}[1]{\textcolor[rgb]{0.00,0.00,0.00}{#1}}
\newcommand{\ControlFlowTok}[1]{\textcolor[rgb]{0.13,0.29,0.53}{\textbf{#1}}}
\newcommand{\DataTypeTok}[1]{\textcolor[rgb]{0.13,0.29,0.53}{#1}}
\newcommand{\DecValTok}[1]{\textcolor[rgb]{0.00,0.00,0.81}{#1}}
\newcommand{\DocumentationTok}[1]{\textcolor[rgb]{0.56,0.35,0.01}{\textbf{\textit{#1}}}}
\newcommand{\ErrorTok}[1]{\textcolor[rgb]{0.64,0.00,0.00}{\textbf{#1}}}
\newcommand{\ExtensionTok}[1]{#1}
\newcommand{\FloatTok}[1]{\textcolor[rgb]{0.00,0.00,0.81}{#1}}
\newcommand{\FunctionTok}[1]{\textcolor[rgb]{0.00,0.00,0.00}{#1}}
\newcommand{\ImportTok}[1]{#1}
\newcommand{\InformationTok}[1]{\textcolor[rgb]{0.56,0.35,0.01}{\textbf{\textit{#1}}}}
\newcommand{\KeywordTok}[1]{\textcolor[rgb]{0.13,0.29,0.53}{\textbf{#1}}}
\newcommand{\NormalTok}[1]{#1}
\newcommand{\OperatorTok}[1]{\textcolor[rgb]{0.81,0.36,0.00}{\textbf{#1}}}
\newcommand{\OtherTok}[1]{\textcolor[rgb]{0.56,0.35,0.01}{#1}}
\newcommand{\PreprocessorTok}[1]{\textcolor[rgb]{0.56,0.35,0.01}{\textit{#1}}}
\newcommand{\RegionMarkerTok}[1]{#1}
\newcommand{\SpecialCharTok}[1]{\textcolor[rgb]{0.00,0.00,0.00}{#1}}
\newcommand{\SpecialStringTok}[1]{\textcolor[rgb]{0.31,0.60,0.02}{#1}}
\newcommand{\StringTok}[1]{\textcolor[rgb]{0.31,0.60,0.02}{#1}}
\newcommand{\VariableTok}[1]{\textcolor[rgb]{0.00,0.00,0.00}{#1}}
\newcommand{\VerbatimStringTok}[1]{\textcolor[rgb]{0.31,0.60,0.02}{#1}}
\newcommand{\WarningTok}[1]{\textcolor[rgb]{0.56,0.35,0.01}{\textbf{\textit{#1}}}}
\usepackage{graphicx}
\makeatletter
\def\maxwidth{\ifdim\Gin@nat@width>\linewidth\linewidth\else\Gin@nat@width\fi}
\def\maxheight{\ifdim\Gin@nat@height>\textheight\textheight\else\Gin@nat@height\fi}
\makeatother
% Scale images if necessary, so that they will not overflow the page
% margins by default, and it is still possible to overwrite the defaults
% using explicit options in \includegraphics[width, height, ...]{}
\setkeys{Gin}{width=\maxwidth,height=\maxheight,keepaspectratio}
% Set default figure placement to htbp
\makeatletter
\def\fps@figure{htbp}
\makeatother
\setlength{\emergencystretch}{3em} % prevent overfull lines
\providecommand{\tightlist}{%
  \setlength{\itemsep}{0pt}\setlength{\parskip}{0pt}}
\setcounter{secnumdepth}{5}
\usepackage{helvet} \renewcommand\familydefault{\sfdefault}
\usepackage{booktabs}
\usepackage{longtable}
\usepackage{array}
\usepackage{multirow}
\usepackage{wrapfig}
\usepackage{float}
\usepackage{colortbl}
\usepackage{pdflscape}
\usepackage{tabu}
\usepackage{threeparttable}
\usepackage{threeparttablex}
\usepackage[normalem]{ulem}
\usepackage{makecell}
\usepackage{xcolor}
\ifLuaTeX
  \usepackage{selnolig}  % disable illegal ligatures
\fi
\IfFileExists{bookmark.sty}{\usepackage{bookmark}}{\usepackage{hyperref}}
\IfFileExists{xurl.sty}{\usepackage{xurl}}{} % add URL line breaks if available
\urlstyle{same} % disable monospaced font for URLs
\hypersetup{
  pdftitle={Prova 3},
  pdfauthor={César Galvão - 19/0011572},
  hidelinks,
  pdfcreator={LaTeX via pandoc}}

\title{Prova 3}
\author{César Galvão - 19/0011572}
\date{2022-09-21}

\begin{document}
\maketitle

\newpage{}

{
\setcounter{tocdepth}{3}
\tableofcontents
}
\let\oldsection\section
\renewcommand\section{\clearpage\oldsection}

\hypertarget{questuxe3o-1}{%
\section{Questão 1}\label{questuxe3o-1}}

\hypertarget{modelo-e-estimativas}{%
\subsection{Modelo e estimativas}\label{modelo-e-estimativas}}

Utiliza-se o modelo de experimento fatorial com parcela dividida,
expresso da seguinte forma

\begin{align}
  y_{ijk} = \mu + \tau_i + \beta_j + (\tau\beta)_{ij} + \gamma_k + (\beta\gamma)_{jk} + \epsilon_{ijk}
\end{align}

Em que \(\tau_i\) representa o efeito do bloco ou repetição (dia) \(i\),
\(\beta_j\) o efeito de parcela do fator Temperatura,
\((\tau\beta)_{ij}\) o resíduo da parcela Temperatura, \(\gamma_k\) o
efeito de subparcela do fator Pressão, \((\beta\gamma)_{jk}\) a
interação entre Pressão e Temperatura e, finalmente, \(\epsilon_{ijk}\)
é o resíduo.

Nas tabelas de estimadores a seguir, a variância é o quadrado médio do
resíduo combinado, que é uma ponderação entre os quadrados médios dentro
e entre blocos. Ponderação semelhante é feita para os graus de liberdade
do resíduo combinado.

\begin{longtable}{cccccccc}
\toprule
$\bar{X}$ & $S^2$ & $\beta_1$ & $\beta_2$ & $\beta_3$ & $\gamma_1$ & $\gamma_2$ & $\gamma_3$\\
\midrule
\endfirsthead
\multicolumn{8}{@{}l}{\textit{(continued)}}\\
\toprule
$\bar{X}$ & $S^2$ & $\beta_1$ & $\beta_2$ & $\beta_3$ & $\gamma_1$ & $\gamma_2$ & $\gamma_3$\\
\midrule
\endhead

\endfoot
\bottomrule
\endlastfoot
\cellcolor{gray!15}{88.7944} & \cellcolor{gray!15}{0.916} & \cellcolor{gray!15}{85.7833} & \cellcolor{gray!15}{89.0667} & \cellcolor{gray!15}{91.5333} & \cellcolor{gray!15}{88.5167} & \cellcolor{gray!15}{88.3} & \cellcolor{gray!15}{89.5667}\\*
\end{longtable}

\begin{longtable}{cc}
\toprule
interacoes & medias\\
\midrule
\endfirsthead
\multicolumn{2}{@{}l}{\textit{(continued)}}\\
\toprule
interacoes & medias\\
\midrule
\endhead

\endfoot
\bottomrule
\endlastfoot
\cellcolor{gray!15}{lo 250} & \cellcolor{gray!15}{\vphantom{1} 86.20}\\
mid 250 & \vphantom{1} 84.60\\
\cellcolor{gray!15}{hi 250} & \cellcolor{gray!15}{\vphantom{1} 86.55}\\
lo 260 & \vphantom{1} 88.95\\
\cellcolor{gray!15}{mid 260} & \cellcolor{gray!15}{\vphantom{1} 88.60}\\
hi 260 & \vphantom{1} 89.65\\
\cellcolor{gray!15}{lo 270} & \cellcolor{gray!15}{\vphantom{1} 90.40}\\
mid 270 & \vphantom{1} 91.70\\
\cellcolor{gray!15}{hi 270} & \cellcolor{gray!15}{\vphantom{1} 92.50}\\
lo 250 & 86.20\\
\cellcolor{gray!15}{mid 250} & \cellcolor{gray!15}{84.60}\\
hi 250 & 86.55\\
\cellcolor{gray!15}{lo 260} & \cellcolor{gray!15}{88.95}\\
mid 260 & 88.60\\
\cellcolor{gray!15}{hi 260} & \cellcolor{gray!15}{89.65}\\
lo 270 & 90.40\\
\cellcolor{gray!15}{mid 270} & \cellcolor{gray!15}{91.70}\\
hi 270 & 92.50\\*
\end{longtable}

\hypertarget{anova}{%
\subsection{ANOVA}\label{anova}}

A tabela de análise variância é apresentada a seguir. A primeira
hipótese testada é se há, na estrutura de temperatura/bloco ou
temperatura/dia, diferença entre nos níveis de temperatura. A segunda
hipótese testada ocorre na estrutura dentro dos blocos/dias, na qual se
avalia se há efeito significativo do fator pressão e se há efeito
significativo na interação entre pressão e temperatura dentro dos
blocos.

\begin{Shaded}
\begin{Highlighting}[]
\FunctionTok{summary}\NormalTok{(modelo)}
\end{Highlighting}
\end{Shaded}

\begin{verbatim}
## 
## Error: dia
##           Df Sum Sq Mean Sq F value Pr(>F)
## Residuals  1  13.01   13.01               
## 
## Error: dia:temp
##           Df Sum Sq Mean Sq F value Pr(>F)  
## temp       2  99.85   49.93   39.26 0.0248 *
## Residuals  2   2.54    1.27                 
## ---
## Signif. codes:  0 '***' 0.001 '**' 0.01 '*' 0.05 '.' 0.1 ' ' 1
## 
## Error: Within
##              Df Sum Sq Mean Sq F value Pr(>F)  
## pressao       1  3.308   3.308   4.480 0.0634 .
## pressao:temp  2  1.715   0.857   1.162 0.3558  
## Residuals     9  6.644   0.738                 
## ---
## Signif. codes:  0 '***' 0.001 '**' 0.01 '*' 0.05 '.' 0.1 ' ' 1
\end{verbatim}

De fato, para a divisão em parcelas do experimento, parece haver efeito
do fator principal temperatura, dado seu pvalor de 0,0248. Para a
divisão em subparcelas, nenhum dos fatores parece ter efeito
significativo. No entanto, pressão tem p-valor próximo a 0,05, o que
poderia sinalizar a necessidade de avaliação mais acurada desse fator em
uma etapa subsequente.

\hypertarget{erro-tipo-2}{%
\subsection{Erro tipo 2}\label{erro-tipo-2}}

Calcula-se a probabilidade de erro tipo 2 considerando:

\begin{itemize}
\tightlist
\item
  \(\tau_i = \{0,1,-1\}, i = 1, 2, 3\);
\item
  \(n = 1, a = 3, b = 3\);
\item
  \(\phi^2 = n\cdot a\cdot b \cdot \frac{\sum \tau^2}{QM_\text{res Comb}}\);
\item
  \(gl_{comb} = \frac{[QMres_A + (k+1)QMres_B]^2}{\frac{QMres_A^2}{gl res_A} + \frac{QMres_B^2}{gl res_B}}\)
\end{itemize}

Obtém-se erro tipo II igual a 0.399 utilizando a distribuição F com
graus de liberdade \((a-1)(b-1)\) e \(gl_{comb}\) e \(\phi^2\) como
parâmetro de não centralidade.

\hypertarget{questuxe3o-2}{%
\section{Questão 2}\label{questuxe3o-2}}

\begin{Shaded}
\begin{Highlighting}[]
\NormalTok{y }\OtherTok{\textless{}{-}} \FunctionTok{c}\NormalTok{(}\DecValTok{107}\NormalTok{,}\DecValTok{89}\NormalTok{,}\DecValTok{116}\NormalTok{,}\DecValTok{101}\NormalTok{,}\DecValTok{90}\NormalTok{,}\DecValTok{96}\NormalTok{, }\CommentTok{\#por linha}
\DecValTok{117}\NormalTok{,}\DecValTok{101}\NormalTok{,}\DecValTok{136}\NormalTok{,}\DecValTok{110}\NormalTok{,}\DecValTok{112}\NormalTok{,}\DecValTok{89}\NormalTok{,}
\DecValTok{122}\NormalTok{,}\DecValTok{98}\NormalTok{,}\DecValTok{139}\NormalTok{,}\DecValTok{104}\NormalTok{,}\DecValTok{99}\NormalTok{,}\DecValTok{92}\NormalTok{,}
\DecValTok{111}\NormalTok{,}\DecValTok{101}\NormalTok{,}\DecValTok{122}\NormalTok{,}\DecValTok{91}\NormalTok{,}\DecValTok{105}\NormalTok{,}\DecValTok{78}\NormalTok{,}
\DecValTok{90}\NormalTok{,}\DecValTok{95}\NormalTok{,}\DecValTok{117}\NormalTok{,}\DecValTok{100}\NormalTok{,}\DecValTok{110}\NormalTok{,}\DecValTok{90}\NormalTok{,}
\DecValTok{116}\NormalTok{,}\DecValTok{90}\NormalTok{,}\DecValTok{114}\NormalTok{,}\DecValTok{94}\NormalTok{,}\DecValTok{114}\NormalTok{,}\DecValTok{93}\NormalTok{)}

\NormalTok{blocos }\OtherTok{\textless{}{-}} \FunctionTok{factor}\NormalTok{(}\FunctionTok{rep}\NormalTok{(}\FunctionTok{c}\NormalTok{(}\StringTok{"I"}\NormalTok{, }\StringTok{"II"}\NormalTok{, }\StringTok{"III"}\NormalTok{, }\StringTok{"IV"}\NormalTok{, }\StringTok{"V"}\NormalTok{, }\StringTok{"VI"}\NormalTok{), }\AttributeTok{each =} \DecValTok{6}\NormalTok{))}

\NormalTok{racao }\OtherTok{\textless{}{-}} \FunctionTok{rep}\NormalTok{(}\FunctionTok{rep}\NormalTok{(}\FunctionTok{c}\NormalTok{(}\StringTok{"A"}\NormalTok{, }\StringTok{"B"}\NormalTok{, }\StringTok{"C"}\NormalTok{), }\AttributeTok{each =} \DecValTok{2}\NormalTok{),}\DecValTok{6}\NormalTok{)}

\NormalTok{suplemento }\OtherTok{\textless{}{-}} \FunctionTok{rep}\NormalTok{(}\FunctionTok{c}\NormalTok{(}\StringTok{"M"}\NormalTok{, }\StringTok{"P"}\NormalTok{), }\DecValTok{18}\NormalTok{)}

\NormalTok{dados2 }\OtherTok{\textless{}{-}} \FunctionTok{data.frame}\NormalTok{(y, blocos, racao, suplemento)}

\NormalTok{modelo }\OtherTok{\textless{}{-}} \FunctionTok{aov}\NormalTok{(y }\SpecialCharTok{\textasciitilde{}}\NormalTok{ racao}\SpecialCharTok{*}\NormalTok{suplemento }\SpecialCharTok{+} \FunctionTok{Error}\NormalTok{(blocos}\SpecialCharTok{/}\NormalTok{racao))}
\end{Highlighting}
\end{Shaded}

\hypertarget{anuxe1lise-do-experimento-e-descriuxe7uxe3o-dos-componentes}{%
\subsection{Análise do experimento e descrição dos
componentes}\label{anuxe1lise-do-experimento-e-descriuxe7uxe3o-dos-componentes}}

Trata-se de novamente de um modelo de efeitos em parcelas divididas.
Existe uma restrição da casualização dada por blocos \(\tau_i\), três
tratamentos de parcela \(\beta_j\) (rações) e dois tratamentos de
subparcela \(\gamma_k\) (suplementos). A interação entre suplemento e
ração é representada por \((\beta\gamma)_{jk}\).

\begin{align}
  y_{ijk} = \mu + \tau_i + \beta_j + (\tau\beta)_{ij} + \gamma_k + (\beta\gamma)_{jk} + \epsilon_{ijk}
\end{align}

As hipóteses testadas na análise de variância são análogas àquelas do
exercício 1: a primeira hipótese testada é se há, na estrutura de
suplemento/bloco diferença entre nos níveis de suplemento. A segunda
hipótese testada ocorre na estrutura dentro dos blocos/racao, na qual se
avalia se há efeito significativo do fator racao e se há efeito
significativo na interação entre ração e suplemento dentro dos blocos.

Pela tabela de análise de variância, é significativa a diferença entre
os tratamentos principais de ração. Além disso, é significativa também a
diferença entre os tipos de suplementação.

\begin{Shaded}
\begin{Highlighting}[]
\FunctionTok{summary}\NormalTok{(modelo)}
\end{Highlighting}
\end{Shaded}

\begin{verbatim}
## 
## Error: blocos
##           Df Sum Sq Mean Sq F value Pr(>F)
## Residuals  5  658.5   131.7               
## 
## Error: blocos:racao
##           Df Sum Sq Mean Sq F value  Pr(>F)   
## racao      2 1310.7   655.4   9.601 0.00471 **
## Residuals 10  682.6    68.3                   
## ---
## Signif. codes:  0 '***' 0.001 '**' 0.01 '*' 0.05 '.' 0.1 ' ' 1
## 
## Error: Within
##                  Df Sum Sq Mean Sq F value   Pr(>F)    
## suplemento        1 2934.0  2934.0  50.874 3.43e-06 ***
## racao:suplemento  2  159.4    79.7   1.382    0.281    
## Residuals        15  865.1    57.7                     
## ---
## Signif. codes:  0 '***' 0.001 '**' 0.01 '*' 0.05 '.' 0.1 ' ' 1
\end{verbatim}

\hypertarget{tukey}{%
\subsection{Tukey}\label{tukey}}

Calcula-se novamente o QMRES combinado e os graus de liberdade
combinados. Supondo exclusivamente o tipo M de suplemento, utiliza-se o
modelo
\texttt{aov(y\ \textasciitilde{}\ racao\ +\ blocos,\ data\ =\ dados\_m)}
como input do teste de Tukey, que tem como hipótese nula que a distância
entre os grupos testados não é significativamente diferente. As
estatísticas são expostas a seguir, sugerindo que as rações C e A não
diferem entre si, mas que B é diferente tanto de B quanto de A

\begin{Shaded}
\begin{Highlighting}[]
\NormalTok{qmres\_comb }\OtherTok{\textless{}{-}}\NormalTok{ (}\FunctionTok{tidy}\NormalTok{(modelo)}\SpecialCharTok{$}\NormalTok{meansq[}\DecValTok{3}\NormalTok{]}\SpecialCharTok{+}\FunctionTok{tidy}\NormalTok{(modelo)}\SpecialCharTok{$}\NormalTok{meansq[}\DecValTok{6}\NormalTok{])}\SpecialCharTok{/}\DecValTok{2}

\NormalTok{glcomb }\OtherTok{\textless{}{-}}\NormalTok{ ((}\FunctionTok{tidy}\NormalTok{(modelo)}\SpecialCharTok{$}\NormalTok{meansq[}\DecValTok{3}\NormalTok{]}\SpecialCharTok{+}\FunctionTok{tidy}\NormalTok{(modelo)}\SpecialCharTok{$}\NormalTok{meansq[}\DecValTok{6}\NormalTok{])}\SpecialCharTok{\^{}}\DecValTok{2}\NormalTok{)}\SpecialCharTok{/} \CommentTok{\#numerador}
\NormalTok{  ((}\FunctionTok{tidy}\NormalTok{(modelo)}\SpecialCharTok{$}\NormalTok{meansq[}\DecValTok{3}\NormalTok{]}\SpecialCharTok{\^{}}\DecValTok{2}\SpecialCharTok{/}\DecValTok{10}\NormalTok{)}\SpecialCharTok{+}\NormalTok{(}\FunctionTok{tidy}\NormalTok{(modelo)}\SpecialCharTok{$}\NormalTok{meansq[}\DecValTok{6}\NormalTok{]}\SpecialCharTok{\^{}}\DecValTok{2}\SpecialCharTok{/}\DecValTok{15}\NormalTok{))}\CommentTok{\#denominador}

\FunctionTok{qtukey}\NormalTok{(.}\DecValTok{95}\NormalTok{, }\DecValTok{2}\NormalTok{,glcomb)}\SpecialCharTok{*}\FunctionTok{sqrt}\NormalTok{(qmres\_comb}\SpecialCharTok{/}\DecValTok{2}\NormalTok{)}
\end{Highlighting}
\end{Shaded}

\begin{verbatim}
## [1] 16.4127
\end{verbatim}

\begin{Shaded}
\begin{Highlighting}[]
\NormalTok{dados\_m }\OtherTok{\textless{}{-}}\NormalTok{ dados2 }\SpecialCharTok{\%\textgreater{}\%} \FunctionTok{filter}\NormalTok{(suplemento }\SpecialCharTok{==} \StringTok{"M"}\NormalTok{)}

\CommentTok{\# tapply(y[suplemento == "M"], racao[suplemento == "M"], mean)}
\CommentTok{\# }
\CommentTok{\# dist(tapply(y[suplemento == "M"], racao[suplemento == "M"], mean))}

\NormalTok{modelo }\OtherTok{\textless{}{-}} \FunctionTok{aov}\NormalTok{(y }\SpecialCharTok{\textasciitilde{}}\NormalTok{ racao }\SpecialCharTok{+}\NormalTok{ blocos, }\AttributeTok{data =}\NormalTok{ dados\_m)}

\FunctionTok{TukeyHSD}\NormalTok{(modelo) }\SpecialCharTok{\%\textgreater{}\%} \FunctionTok{tidy}\NormalTok{() }\SpecialCharTok{\%\textgreater{}\%}
  \FunctionTok{filter}\NormalTok{(term }\SpecialCharTok{==} \StringTok{"racao"}\NormalTok{)}\SpecialCharTok{\%\textgreater{}\%}
  \FunctionTok{select}\NormalTok{(}\SpecialCharTok{{-}}\StringTok{\textasciigrave{}}\AttributeTok{null.value}\StringTok{\textasciigrave{}}\NormalTok{)}
\end{Highlighting}
\end{Shaded}

\begin{verbatim}
## # A tibble: 3 x 6
##   term  contrast estimate conf.low conf.high adj.p.value
##   <chr> <chr>       <dbl>    <dbl>     <dbl>       <dbl>
## 1 racao B-A         13.5     -1.28     28.3       0.0736
## 2 racao C-A         -5.50   -20.3       9.28      0.582 
## 3 racao C-B        -19      -33.8      -4.22      0.0139
\end{verbatim}

\hypertarget{maximizauxe7uxe3o-da-resposta}{%
\subsection{Maximização da
resposta}\label{maximizauxe7uxe3o-da-resposta}}

Com base exclusivamente no teste de Tukey e as médias para cada ração,
expostas na tabela abaixo, opta-se pela Ração B, já que possui maior
média e é a única que testou como significativamente diferente das
demais.

\begin{Shaded}
\begin{Highlighting}[]
\NormalTok{dados\_m }\SpecialCharTok{\%\textgreater{}\%} \FunctionTok{group\_by}\NormalTok{(racao) }\SpecialCharTok{\%\textgreater{}\%} \FunctionTok{summarise}\NormalTok{(}\AttributeTok{media =} \FunctionTok{mean}\NormalTok{(y))}
\end{Highlighting}
\end{Shaded}

\begin{verbatim}
## # A tibble: 3 x 2
##   racao media
##   <chr> <dbl>
## 1 A      110.
## 2 B      124 
## 3 C      105
\end{verbatim}

\end{document}
