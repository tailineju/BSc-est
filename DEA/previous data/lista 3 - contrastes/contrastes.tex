% Options for packages loaded elsewhere
\PassOptionsToPackage{unicode}{hyperref}
\PassOptionsToPackage{hyphens}{url}
%
\documentclass[
]{article}
\usepackage{amsmath,amssymb}
\usepackage{lmodern}
\usepackage{iftex}
\ifPDFTeX
  \usepackage[T1]{fontenc}
  \usepackage[utf8]{inputenc}
  \usepackage{textcomp} % provide euro and other symbols
\else % if luatex or xetex
  \usepackage{unicode-math}
  \defaultfontfeatures{Scale=MatchLowercase}
  \defaultfontfeatures[\rmfamily]{Ligatures=TeX,Scale=1}
\fi
% Use upquote if available, for straight quotes in verbatim environments
\IfFileExists{upquote.sty}{\usepackage{upquote}}{}
\IfFileExists{microtype.sty}{% use microtype if available
  \usepackage[]{microtype}
  \UseMicrotypeSet[protrusion]{basicmath} % disable protrusion for tt fonts
}{}
\makeatletter
\@ifundefined{KOMAClassName}{% if non-KOMA class
  \IfFileExists{parskip.sty}{%
    \usepackage{parskip}
  }{% else
    \setlength{\parindent}{0pt}
    \setlength{\parskip}{6pt plus 2pt minus 1pt}}
}{% if KOMA class
  \KOMAoptions{parskip=half}}
\makeatother
\usepackage{xcolor}
\IfFileExists{xurl.sty}{\usepackage{xurl}}{} % add URL line breaks if available
\IfFileExists{bookmark.sty}{\usepackage{bookmark}}{\usepackage{hyperref}}
\hypersetup{
  pdftitle={Exercício de laboratorio 3},
  pdfauthor={César A. Galvão - 19/0011572},
  hidelinks,
  pdfcreator={LaTeX via pandoc}}
\urlstyle{same} % disable monospaced font for URLs
\usepackage[margin=1in]{geometry}
\usepackage{graphicx}
\makeatletter
\def\maxwidth{\ifdim\Gin@nat@width>\linewidth\linewidth\else\Gin@nat@width\fi}
\def\maxheight{\ifdim\Gin@nat@height>\textheight\textheight\else\Gin@nat@height\fi}
\makeatother
% Scale images if necessary, so that they will not overflow the page
% margins by default, and it is still possible to overwrite the defaults
% using explicit options in \includegraphics[width, height, ...]{}
\setkeys{Gin}{width=\maxwidth,height=\maxheight,keepaspectratio}
% Set default figure placement to htbp
\makeatletter
\def\fps@figure{htbp}
\makeatother
\setlength{\emergencystretch}{3em} % prevent overfull lines
\providecommand{\tightlist}{%
  \setlength{\itemsep}{0pt}\setlength{\parskip}{0pt}}
\setcounter{secnumdepth}{5}
\usepackage{helvet} \renewcommand\familydefault{\sfdefault}
\usepackage{booktabs}
\usepackage{longtable}
\usepackage{array}
\usepackage{multirow}
\usepackage{wrapfig}
\usepackage{float}
\usepackage{colortbl}
\usepackage{pdflscape}
\usepackage{tabu}
\usepackage{threeparttable}
\usepackage{threeparttablex}
\usepackage[normalem]{ulem}
\usepackage{makecell}
\usepackage{xcolor}
\ifLuaTeX
  \usepackage{selnolig}  % disable illegal ligatures
\fi

\title{Exercício de laboratorio 3}
\usepackage{etoolbox}
\makeatletter
\providecommand{\subtitle}[1]{% add subtitle to \maketitle
  \apptocmd{\@title}{\par {\large #1 \par}}{}{}
}
\makeatother
\subtitle{Contrastes}
\author{César A. Galvão - 19/0011572}
\date{2022-07-06}

\begin{document}
\maketitle

\newpage{}

{
\setcounter{tocdepth}{2}
\tableofcontents
}
\let\oldsection\section
\renewcommand\section{\clearpage\oldsection}

\hypertarget{questao-1}{%
\section{Questao 1}\label{questao-1}}

\begin{longtable}{cccc}
\toprule
Químico &  & \% de álcool metílico & \\
\midrule
\endfirsthead
\multicolumn{4}{@{}l}{\textit{(continued)}}\\
\toprule
Químico &  & \% de álcool metílico & \\
\midrule
\endhead

\endfoot
\bottomrule
\endlastfoot
\cellcolor{gray!15}{I} & \cellcolor{gray!15}{84.99} & \cellcolor{gray!15}{84.04} & \cellcolor{gray!15}{84.38}\\
II & 85.15 & 85.13 & 84.88\\
\cellcolor{gray!15}{III} & \cellcolor{gray!15}{84.72} & \cellcolor{gray!15}{84.48} & \cellcolor{gray!15}{85.16}\\
IV & 84.20 & 84.10 & 84.55\\*
\end{longtable}

\hypertarget{analise-o-experimento-para-avaliar-se-os-funcionuxe1rios-diferem-significativamente.-apresente-o-modelo-as-hipuxf3teses-e-a-tabela-da-anuxe1lise-de-variuxe2ncia.-use-alfa-005.}{%
\subsection{Analise o experimento para avaliar se os funcionários
diferem significativamente. Apresente o modelo, as hipóteses e a tabela
da análise de variância. Use alfa =
0,05.}\label{analise-o-experimento-para-avaliar-se-os-funcionuxe1rios-diferem-significativamente.-apresente-o-modelo-as-hipuxf3teses-e-a-tabela-da-anuxe1lise-de-variuxe2ncia.-use-alfa-005.}}

A comparação das médias dos grupos será realizada mediante análise de
variância. O modelo escolhido para tal é o modelo de efeitos, expresso
na equação a seguir

\begin{align*}
  y_{ij} = \mu + \tau_i + e_{ij}, \quad i = 1, 2,..., a; \quad j = 1, 2,..., n
\end{align*}

em que \(\mu\) é a média geral, \(\tau_i\) é a média ou efeito dos
grupos -- cada químico sendo considerado um tratamento -- e \(e_{ij}\) é
o desvio do elemento. Os grupos são indexados por \(i\) e os indivíduos
de cada grupo indexados por \(j\).

As hipóteses do teste são as seguintes: \begin{align*}
  \begin{cases}
    H_0: \tau_1 = ... = \tau_a = 0, \quad \text{(O efeito de tratamento é nulo)}\\
    H_1: \exists \tau_i \neq 0
  \end{cases}
\end{align*}

que equivale dizer

\begin{align*}
  \begin{cases}
    H_0: \mu_1 = ... = \mu_a\\
    H_1: \exists \mu_i \neq \mu_j, \, i \neq j.
  \end{cases}
\end{align*}

Neste exercício, pressupõe-se normalidade dos dados -- e
consequentemente dos resíduos -- e igualdade de variâncias. Não sendo
necessário proceder com os testes diagnósticos, apresenta-se tabela de
análise de variância a seguir:

\begin{longtable}{lccccl}
\toprule
Fonte de variação & g.l. & SQ & MQ & Estatística F & p-valor\\
\midrule
\endfirsthead
\multicolumn{6}{@{}l}{\textit{(continued)}}\\
\toprule
Fonte de variação & g.l. & SQ & MQ & Estatística F & p-valor\\
\midrule
\endhead

\endfoot
\bottomrule
\endlastfoot
\cellcolor{gray!15}{chemist} & \cellcolor{gray!15}{3} & \cellcolor{gray!15}{1.0446} & \cellcolor{gray!15}{0.3482} & \cellcolor{gray!15}{3.2458} & \cellcolor{gray!15}{0.0813}\\
Residuals & 8 & 0.8582 & 0.1073 & NA & NA\\*
\end{longtable}

Considerando \(\alpha = 0,05\) não se rejeita a hipótese nula. Ou seja,
não se pode dizer que há um químico cuja média de percentual de álcool
metílico é diferente dos demais.

\hypertarget{considere-que-o-funcionuxe1rio-2-uxe9-novo-na-empresa-e-construa-um-conjunto-de-contrastes-ortogonais-a-partir-dessa-informauxe7uxe3o.-apresente-as-hipuxf3teses-que-seruxe3o-testadas-as-conclusuxf5es-e-a-estatuxedstica-de-teste-considerada.}{%
\subsection{Considere que o funcionário 2 é novo na empresa e construa
um conjunto de contrastes ortogonais a partir dessa informação.
Apresente as hipóteses que serão testadas, as conclusões e a estatística
de teste
considerada.}\label{considere-que-o-funcionuxe1rio-2-uxe9-novo-na-empresa-e-construa-um-conjunto-de-contrastes-ortogonais-a-partir-dessa-informauxe7uxe3o.-apresente-as-hipuxf3teses-que-seruxe3o-testadas-as-conclusuxf5es-e-a-estatuxedstica-de-teste-considerada.}}

Compararemos dois subgrupos, formados a partir do conjunto inicial de
tratamentos, para realizar o teste de comparação de médias utilizando
contrastes. A saber, compararemos a média das medidas do químico 2 com a
média dos demais. Construimos os seguintes contrastes:

\begin{align}
  &\Gamma_1 = \sum\limits_{i = 1}^{a} c_i \mu_i \quad \text{em que} \sum\limits_{i = 1}^{4} c_i = 0; \text{ e } c_i = \left\{-\frac{1}{3}, 1, -\frac{1}{3}, -\frac{1}{3}\right\}
\end{align}

Para a construção dos demais contrastes ortogonais, fazemos

\begin{align*}
  &\Gamma_2 = \sum\limits_{i = 1}^{3} c_i \mu_i \quad \longrightarrow
  c_i = \left\{1, 0, -\frac{1}{2}, -\frac{1}{2}\right\} \\
  &\Gamma_3 = \sum\limits_{i = 1}^{2} c_i \mu_i \quad \longrightarrow
  c_i = \left\{0, 0, 1, -1\right\}
\end{align*}

de modo que todos os \(c_i, i = 1, 2, 3\) são ortogonais entre si. Dessa
forma, as hipóteses testadas são as seguintes:

\begin{align*}
  \text{Contraste 1: }&\begin{cases}
    H_0: \mu_2 = \frac{\mu_1+\mu_3+\mu_4}{3}\\
    H_1: \mu_2 \neq \frac{\mu_1+\mu_3+\mu_4}{3}
  \end{cases}\\
  \text{Contraste 2: }&\begin{cases}
    H_0: \mu_1 = \frac{\mu_3+\mu_4}{2}\\
    H_1: \mu_2 \neq \frac{\mu_3+\mu_4}{2}
  \end{cases} 
  \text{ e }\\
  \text{Contraste 3: }&\begin{cases}
    H_0: \mu_3 = \mu_4\\
    H_1: \mu_3 \neq \mu_4
  \end{cases}
\end{align*}

A estatística de teste para a realização dos contrastes é definida
conforme a expressão a seguir, em que \(\text{QMRES}\) é a soma de
quadrados dos resíduos da ANOVA exposta anteriormente:

\begin{align}
  \frac{\left( \sum\limits_{i = 1}^n c_i \, \bar{y}_{i.}\right)^2}{ \sum\limits_{i = 1}^n c_i^2 \, \frac{\text{QMRES}}{n}} \sim F(1, an-a = 8)
\end{align}

Além disso, considera-se

\begin{align}
  \frac{\left( \sum\limits_{i = 1}^n c_i \, \bar{y}_{i.}\right)^2}{\sum\limits_{i = 1}^n c_i^2} = \frac{\text{SQContraste}_i}{1 \, (g.l.)} = \text{QMContraste}_i
\end{align}

tal que, se os contrastes forem calculados da forma correta, a soma dos
quadrados médios dos contrastes deve ser igual ao quadrado médio dos
tratamentos.

As estatísticas são expostas na tabela de análise de variância a seguir,
decomposta em seus contrastes.

\begin{longtable}{lccccl}
\toprule
Fonte de variação & g.l. & SQ & MQ & Estatística F & p-valor\\
\midrule
\endfirsthead
\multicolumn{6}{@{}l}{\textit{(continued)}}\\
\toprule
Fonte de variação & g.l. & SQ & MQ & Estatística F & p-valor\\
\midrule
\endhead

\endfoot
\bottomrule
\endlastfoot
\cellcolor{gray!15}{chemist} & \cellcolor{gray!15}{3} & \cellcolor{gray!15}{1.0446} & \cellcolor{gray!15}{0.3482} & \cellcolor{gray!15}{3.2458} & \cellcolor{gray!15}{0.0813}\\
C1 & 1 & 0.2187 & 0.2187 & 6.1161 & 0.0385\\
\cellcolor{gray!15}{C2} & \cellcolor{gray!15}{1} & \cellcolor{gray!15}{0.0028} & \cellcolor{gray!15}{0.0028} & \cellcolor{gray!15}{0.0788} & \cellcolor{gray!15}{0.7861}\\
C3 & 1 & 0.1267 & 0.1267 & 3.5425 & 0.0966\\
\cellcolor{gray!15}{Residuals} & \cellcolor{gray!15}{8} & \cellcolor{gray!15}{0.8582} & \cellcolor{gray!15}{0.1073} & \cellcolor{gray!15}{NA} & \cellcolor{gray!15}{NA}\\*
\end{longtable}

De fato, sob \(\alpha = 0,05\), só se pode rejeitar a hipótese nula sob
o Contraste 1. Isso significa que, se o químico 2 for comparado aos
demais químicos, sua média de concentração de álcool é estatisticamente
diferente. Além disso, é possível verificar que a soma dos quadrados
médios dos contrastes equivale ao quadrado médio dos tratamentos, o que
confere validade aos cálculos.

\hypertarget{calcule-a-probabilidade-do-erro-tipo-2-considerando-que-a-diferenuxe7a-entre-dois-funcionuxe1rios-seja-de-1-unidade}{%
\subsection{Calcule a probabilidade do erro tipo 2 considerando que a
diferença entre dois funcionários seja de 1
unidade}\label{calcule-a-probabilidade-do-erro-tipo-2-considerando-que-a-diferenuxe7a-entre-dois-funcionuxe1rios-seja-de-1-unidade}}

Desejamos calcular
\(\beta(\tau_1 = 0.5, \, \tau_2 = -0.5, \, \tau_3 = 0, \, \tau_4 = 0)\).
Para isso, utilizaremos \(n = 3\), \(\alpha = 0,05\) e
\(\sigma^2 = \frac{\text{QMRES}}{n}\). A probabilidade será calculada da
seguinte forma:

\begin{align}
  P\left( F_{\text{obs}} < F_\text{crit} \bigg| \phi^2 =  \frac{n}{\sigma^2} \sum\limits_{i=1}^{4} \tau_i^2  \right),
\end{align}

considerando a variância para os resíduos. Portanto,

\begin{align}
  \phi^2 &= \frac{n}{\text{QMRES}} \sum\limits_{i=1}^{4} \tau_i^2
\end{align}

é o parâmetro de não-centralidade (pnc ou, em inglês, \emph{ncp}) da
distribuição \(F\) e, sob \(H_0\), \(\phi^2 = 0\).

O valor
\(F_\text{crit} = F( \gamma = 0,95; gl_1 = 3; gl_2 = 8, \phi^2 = 0)\) é
de 4.066. Considerando \(\phi^2 =\) 13.9828, obtém-se

\begin{align*}
  P\left( F_{\text{obs}} < F_\text{crit} \big| \phi^2 \text{ sob } H_1 \right) &= P\left( F_{\text{obs}} < 4,066 \big| \phi^2 = 13,982 \right) \\
  &= 0,312
\end{align*}

\hypertarget{qual-deve-ser-o-nuxfamero-de-repetiuxe7uxf5es-nesse-experimento-para-que-o-erro-seja-menor-que-5}{%
\subsection{Qual deve ser o número de repetições nesse experimento para
que o erro seja menor que
5\%?}\label{qual-deve-ser-o-nuxfamero-de-repetiuxe7uxf5es-nesse-experimento-para-que-o-erro-seja-menor-que-5}}

Considerando os métodos de cálculo já utilizados, constroi-se a tabela a
seguir:

\begin{longtable}{ccccccc}
\toprule
n & $\phi^2$ & $\phi$ & g.l. & $F_{\text{crit}}$ & $\beta$ & Poder\\
\midrule
\endfirsthead
\multicolumn{7}{@{}l}{\textit{(continued)}}\\
\toprule
n & $\phi^2$ & $\phi$ & g.l. & $F_{\text{crit}}$ & $\beta$ & Poder\\
\midrule
\endhead

\endfoot
\bottomrule
\endlastfoot
\cellcolor{gray!15}{3} & \cellcolor{gray!15}{13.98} & \cellcolor{gray!15}{3.74} & \cellcolor{gray!15}{8} & \cellcolor{gray!15}{4.07} & \cellcolor{gray!15}{0.31} & \cellcolor{gray!15}{0.69}\\
4 & 18.64 & 4.32 & 12 & 3.49 & 0.12 & 0.88\\
\cellcolor{gray!15}{5} & \cellcolor{gray!15}{23.30} & \cellcolor{gray!15}{4.83} & \cellcolor{gray!15}{16} & \cellcolor{gray!15}{3.24} & \cellcolor{gray!15}{0.04} & \cellcolor{gray!15}{0.96}\\*
\end{longtable}

Considerando os valores da tabela, para que o erro tipo II seja menor
que 5\% são necessários 5 repetições para cada tratamento nesse
experimento.

\hypertarget{exercuxedcio-de-simulauxe7uxe3o}{%
\section{Exercício de simulação}\label{exercuxedcio-de-simulauxe7uxe3o}}

Considerando uma diferença já conhecida entre os efeitos de tratamento,
\((\tau_1 = 0.5, \, \tau_2 = -0.5, \, \tau_3 = 0, \, \tau_4 = 0)\),
foram realizadas 1000 iterações da expressão a seguir, visando obter
empiricamente a probabilidade de erro tipo II.

\begin{align}
  y_{ij} &= \mu + \tau_i + \varepsilon_{ij}
\end{align}

em que \(\varepsilon_{ij} \sim N(0, \text{QMRES})\) representa o erro
aleatório, cuja variância e distribuição são os mesmos de \(y_{ij}\).

Sabe-se que há uma diferença entre os tratamentos, motivo pelo qual uma
análise de variância para cada conjunto
\(y_{ij}, i = 1, 2, 3, 4, j = 1,2,3\) deveria apontar p-valor inferior a
0,05. Considera-se portanto a probabilidade desejada como a proporção
dos casos em que não se rejeitaria a hipótese nula, que tem valor 0.314

\end{document}
